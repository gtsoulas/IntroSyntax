\documentclass[a4paper,12pt]{article}
\usepackage[margin=20mm]{geometry}

% Fonts
\usepackage{libertine}
\usepackage{amsmath}

% Packages
\usepackage{forest}
\usepackage{linguex}
\usepackage{soul}
\usepackage{enumitem}
\usepackage{microtype}

\newcommand{\lbl}[1]{\ensuremath{_{\scriptstyle\mathrm{#1}}}}
\renewcommand{\firstrefdash}{}

\begin{document}
\noindent\textbf{Introduction to Syntax: Chapter 5}\par

\vspace{1em}
\begin{enumerate}
	\item \textbf{NP/DP constituency}
    	\begin{enumerate}[label=(\alph*)]
        	\item The \textul{old men}

              \begin{forest} baseline
                [NP, for tree={parent anchor=south, child anchor=north, align=center, base=bottom}
                [AP[A[old]]] [NP[N[men]]]]
              \end{forest}

            \item The \textul{women and men}

              \begin{forest} baseline
                [NP, for tree={parent anchor=south, child anchor=north, align=center, base=bottom}
                [NP[N[women]]][Conj [and]][NP[N[men]]]]
              \end{forest}

            \item The \textul{women and old men}

              \begin{forest} baseline
                [NP, for tree={parent anchor=south, child anchor=north, align=center, base=bottom}
                [NP [N[women]]] [Conj [and]] [NP[AP[A[old]]] [NP[N[men]]]]]
              \end{forest}

\newpage
            \item He is looking for \textul{very old or new editions}
			
            \begin{enumerate}[label=(\roman*)]
            	\item \textbf{Meaning 1:}
                He is looking for very old editions. He is also looking for new editions, but these editions don't need to be very new. The adjunct \textit{very} only modifies \textit{old}.

                \begin{forest} baseline
                  [NP, for tree={parent anchor=south, child anchor=north, align=center, base=bottom}
                  [AP [AP [AdvP [Adv [very]]] [AP[A[old]]]] [Conj
                  [or]] [AP[A[new]]]] [NP[N[editions]]]]
              	\end{forest}


                \item \textbf{Meaning 2:}
                He is looking for very old and very new editions. Here, \textit{very} modifies both \textit{old} and \textit{new}.

                \begin{forest} baseline
                	[NP, for tree={parent anchor=south, child anchor=north, align=center, base=bottom}
                [AP [AdvP [Adv [very]]] [AP[AP[A[old]]][Conj [or]][AP[A[new]]]]] [NP[N[editions]]]]
              	\end{forest}

            \end{enumerate}
        \end{enumerate}

\newpage
	\item[3.] \textbf{Ambiguity}

		\ex.\label{3.1}A cousin of Bill's brother

    	\begin{enumerate}[label=(\roman*)]
        	\item Interpretation A: Tom is Bill's cousin. The person in question is Tom's brother: [A cousin of Bill]'s brother.\\
				  Interpretation B: Charlie is Bill's brother. The person in question is Charlie's cousin: A cousin of [Bill's brother].

            \item \hspace{3em}Interpretation A\hspace{7.5em} Interpretation B\\
             	\begin{forest} baseline
                  [DP, for tree={parent anchor=south, child anchor=north, align=center, base=bottom}
[DP[D[A]][NP[N[cousin]][PP[P[of]][DP[Bill]]]]][D'[D['s]][NP[N[brother]]]]]
              	\end{forest}
                \qquad\qquad\begin{forest} baseline
                [DP, for tree={parent anchor=south, child anchor=north, align=center, base=bottom}
[D[A]][NP[N[cousin]][PP [P[of]]   [DP [DP [Bill]] [D' [D ['s]] [NP [N [brother]]]]]  ] ]]
              	\end{forest}

		\end{enumerate}

        \ex.\label{intB}Whose brother did you meet a cousin of?
		
        \begin{enumerate}
        	\item[(iii)] When we form \ref{intB}, interpretation A is no longer available. The answer to (2) can only be `Bill', not `a cousin of Bill'.

			\item[(iv)] Recall that if you can move of a string of words, it suggests that the string is a constituent. Therefore, \textit{whose brother} is a constituent. Since the answer to \textit{whose} can only be \textit{Bill}, only the structure for Interpretation B allows for \textit{Bill's brother} to be a constituent. Interpretation A is impossible because \textit{Bill's brother} is not a constituent in that structure.

        \end{enumerate}

\newpage
    \item[5.] \textbf{VP constituency}
    \setcounter{ExNo}{0}

	   	\ex.\a.\label{5.1}The girl will [sleep]
        	\b.\label{5.2}The girl will sleep on the couch
            \c.\label{5.3}The girl will sleep on the couch tomorrow

    	\begin{enumerate}[label=(\roman*)]
        	\item
              \begin{forest} baseline
                [TP, for tree={parent anchor=south, child anchor=north, align=center, base=bottom}
                [DP [D [the]] [NP [N [girl]]]]
                [T' [T [will]] [VP [V [sleep]]]]            ]
              \end{forest}

              \textbf{substitution:} The girl will [sleep] and the boy will [do so] too.

              \textbf{ellipsis:} The girl will [sleep] and the boy will \textst{sleep} too.

              note: This basically shows that intransitive verbs are VPs too, even if they only contain one word.

			\item We apply constituency tests to \textit{sleep} and \textit{sleep on the couch}:
            	
				\ex.\a.The girl will [sleep on the couch] and the boy will [do so] too.
                \b.The girl will [sleep on the couch] and the boy will [do so] on the floor.

				Since both tests are grammatical, it shows that we have two VPs: \textit{sleep} and \textit{sleep on the couch}. This means that \ref{5.2} has the following structure:

	            \begin{forest} baseline
                  [TP, for tree={parent anchor=south, child anchor=north, align=center, base=bottom}
                  [DP [D [the]] [NP [N [girl]]]]
                  [T' [T [will]] [VP [VP [V [sleep]]] [PP [P [on]] [DP [D [the]] [NP [N [couch]]]]] ] ]
                  ]
                \end{forest}

         		The PP \textit{on the couch} is an adjunct because it is a sister of a VP rather than a sister of a head V.

            \item Here, we can show that there are three VPs:

			    \ex.\a.The girl will [sleep] on the couch tomorrow and the boy will [do so] on the floor on Sunday.
	                \b.The girl will [sleep on the couch] tomorrow and the boy will [do so] on Sunday.\hspace{\fill}[substitution]                    			

				\ex.The girl will [sleep on the couch tomorrow] and the boy will \textst{sleep on the couch tomorrow} too.\hspace{\fill}[ellipsis]

           The tests above show that \textit{sleep}, \textit{sleep on the couch} and \textit{sleep on the couch tomorrow} are all VPs, and that the PP \textit{on the couch} and DP \textit{tomorrow} are both adjuncts.

             \begin{forest} baseline
               [TP, for tree={parent anchor=south, child anchor=north, align=center, base=bottom}
               [DP [D [the]] [NP [N [girl]]] ]
               [T' [T [will]] [VP [VP [VP [V [sleep]]] [PP [P [on]] [DP [D [the]] [NP [N [couch]]]]] ] [DP [tomorrow]]]]]
            \end{forest}

            \item The tree suggests that \textit{on the couch tomorrow}, \textit{the couch tomorrow} and \textit{couch tomorrow} are constituents. However, no constituency test shows that this is the case, For example:

            \ex.*On the couch tomorrow, the girl will sleep.

            \ex.*The girl will sleep on [it] = [the couch tomorrow].

            Movement doesn't work, and neither does substitution, i.e. \textit{it} can only mean `the couch' but not `the couch tomorrow'.

            \item The structure fails to express the fact that \textit{on the couch} is a constituent; it only shows that \textit{on the couch tomorrow} is a constituent. It also doesn't show that \textit{the couch} is a constituent.

            \item No, for similar reasons to the \textit{whose brother} in question 3 above. According to wh-movement, wh-questioned strings are constituents (p74). However, in this tree, \textit{which couch} is not a constituent, so we will not be able to move \textit{which couch} to the front.

        \end{enumerate}

\newpage
    \item[7.] \textbf{Complements and adjuncts (in DPs)}\\
        For this question, judgements will vary from speaker to speaker. The general point to take away is that a string [NOUN \textit{of}-PP] generally involves an N that takes a PP complement. This means that \textit{one}-substitution must target the entire string [NOUN \textit{of}-PP] and not just NOUN alone. However, other PPs like \textit{from}-PPs are adjuncts, so a string containing [NOUN \textit{from}-PP] would behave differently---\textit{one}-substitution should be able to target just the NOUN, which means that NOUN is an NP (even if it's just one word).
    \setcounter{ExNo}{0}

        \ex.\a.\label{1a}Student of linguistics from Russia\\
        	Since \ref{2c} is bad and we cannot substitute \textit{one} for \textit{student}, we can conclude that \textit{student} is not an NP. However, \ref{2d} shows that \textit{student of linguistics} is an NP. It follows that the PP \textit{of linguistics} is a complement and combines first with the N head \textit{student}. The PP \textit{from Russia} must therefore be an adjunct.
        \b.\label{1b}A professor of linguistics from Russia\\
        	Structurally identical to \ref{1a} with the addition of the indefinite article \textit{a}. Independent constraints concerning the use of the indefinite article prevent us from using \textit{one} substitution here: \textit{*a one from Russia}, although we can easily do \textit{the one from Russia}.
        \b.\label{1c}A textbook of linguistics from Russia\\
        	Similar arguments to \ref{1b}.
        \b.*\label{1d}A knife of linguistics from Russia\\
        	Here, the ungrammaticality is caused by something different. Recall that a head selects for its complement. So while students, professors and textbook can select for complements that denote academic subjects or topics, a knife obviously cannot since there is no semantic dependence between a knife and linguistics. We can, however, easily establish some other type of dependence, e.g. blade material, which works perfectly as a complement \textit{a knife of damascus steel}, which works perfectly. Since \textit{from Russia} is an adjunct that describes a locative origin, it works fine here: \textit{a knife of damascus steel from Russia}.
        \b.*\label{1e}A thief of linguistics from Russia\\
       		Similar reasoning to \ref{1d}, although this sentence is probably slightly less implausible (suppose linguistics was a closely guarded state secret).
        \b.\label{1f}A student/professor/textbook/knife/thief/table from Russia\\
        	As mentioned in \ref{1d}, \textit{from Russia} describes a locative origin, so meaning-wise, there should be few restrictions on its use as an adjunct, unless something cannot logically be from Russia, e.g. \textit{*a planet from Russia} vs. \textit{a planet from a distant galaxy} or something like that.
        \b.*\label{1g}A student from Russia of Linguistics\\
        	We have already established that \textit{of Linguistics} is a complement and \textit{from Russia} an adjunct. Furthermore, our syntactic theory predicts that complements must be closer to heads than adjuncts, and therefore the ungrammaticality of \ref{1g} is predicted because the adjunct intervenes between the head and complement.
        \b.*\label{1h}Professor from Russia of Linguistics\\
        	Same reasoning as \ref{1g}.
\newpage
        \b.\label{1i}A linguistics student from Russia\\
        	This phrase has the same meaning as \ref{1a}. The difference rather than having a noun head and PP complement structure, we have a compound \textit{linguistics student}. The RHHR correctly predicts that this compound is a type of student, not a type of linguistics. Since it's a compound it should behave like a normal noun, i.e. combine with determiners and allow suitable adjuncts:\\
            \begin{forest} baseline
               [DP, for tree={parent anchor=south, child anchor=north, align=center, base=bottom}
               [D [a]] [NP [NP [N (compound) [N [linguistics]] [N [student]]]] [PP [from Russia, triangle]]]]
            \end{forest}
        \b.\label{1j}A Russian linguistics student\\
        	Since adjuncts can be on the left or right, there's no problem having an adjunct on the left, especially since adjectives appear to the left of NPs in English:\\
            \begin{forest} baseline
               [DP, for tree={parent anchor=south, child anchor=north, align=center, base=bottom}
               [D [a]] [NP [AP [A [Russian]]] [NP [linguistics student, triangle]]]]
            \end{forest}
        \b.*\label{1k}A linguistics Russian student\\
        	Since \textit{linguistics student} is a compound it can never be separated by any other element. This property can be clearly seen with other compounds, e.g. \textit{blackbird}. A `pool table made of wood' can be paraphrased as `wooden [pool table]' but not `*pool wooden table'.

    \ex.\a.\label{2a}The Russian one\\
    	This is grammatical because \textit{one} replaces either \textit{linguistics student} or \textit{student of linguistics}, both of which are NPs.
    \b.*\label{2b}The linguistics one\\
    	When \textit{linguistics} is used as a modifier on the left, it forms a compound as seen in \ref{1i} 		 and \ref{1j}. Since each component of the compound is an N not an NP, we don't expect \textit{one}-		substitution to work, as is the case here.\newpage
    \b.??\label{2c}One of linguistics from Russia\\
    	In the introduction to this question we said that there will be some speaker variation here. This is precisely the source of the variation, whether \textit{of}-PPs are treated as complements or adjuncts for that particular speaker. The majority of people will find \ref{2c} bad, but a small number of will find it marginal and an even smaller number (if any at all) will find it perfect.
    \b.\label{2d}One from Russia\\
    	If \textit{one} replaces \textit{student of linguistics}, which is an NP, then the grammaticality here is straightforwardly predicted.

    \ex.\a.\label{3a}An Italian student of math\\
    	Based on our discussion above a construction \textit{student of X} involves \textit{student} taking \textit{of X} as a complement:\\
        \begin{forest} baseline
          [DP, for tree={parent anchor=south, child anchor=north, align=center, base=bottom}
          [D [an]] [NP [AP [A [Italian]]] [NP [N [student]] [PP [P [of]] [NP [N [math]]]]]]]
        \end{forest}
    \b.\label{3b}A math student from Italy\\
    	When \textit{student of X} is is turned into \textit{X student}, then it is a compound. So the structure would be identical to \ref{1i}, substituting \textit{Italian} for \textit{Russian}, and \textit{math} for \textit{linguistics}.
    \b.\label{3c}An Italian math student\\
    	Same as \ref{1j}.
    \b.*\label{3d}An math Italian student\\
    	The use of \textit{an} instead of \textit{a} is possibly a typo and irrelevant here. Otherwise, same explanation as \ref{1k}.

\end{enumerate}

\end{document} 