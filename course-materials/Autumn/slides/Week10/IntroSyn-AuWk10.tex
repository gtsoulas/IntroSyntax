\documentclass[10pt]{beamer}
%{article}
%\usepackage{beamerarticle}
\usepackage{tipa}
\usetheme[
%sectiontitleformat=regular,
%everytitleformat=regular,
block=fill,
sectionpage=progressbar,
progressbar=foot,
numbering=fraction,
background=dark]{metropolis}
 
\setbeamercovered{invisible}

%\usecolortheme{owl}
\setbeamertemplate{blocks}[rounded][shadow=true]
\usepackage{booktabs}
\usepackage[scale=2]{ccicons}

\usepackage{pgfplots}
\usepgfplotslibrary{dateplot}


% Packages
\usepackage{linguexsf}
\usepackage{forest}
\usepackage{forest-animate}
\usetikzlibrary{arrows.meta} 
%\usepackage[usenames, dvipsnames]{color}
\usepackage{amsmath}
\usepackage{centernot}

  \tikzset{
    invisible/.style={opacity=0,text opacity=0},
    visible on/.style={alt=#1{}{invisible}},
    alt/.code args={<#1>#2#3}{\alt<#1>{\pgfkeysalso{#2}}{\pgfkeysalso{#3}}}}
\forestset{
  alert on/.style={
    /tikz/alt=<#1>{node options={/tikz/text=alert}}{}},
  dont show before/.style={
%    for first={alert on=#1},
    for ancestors'={
      /tikz/visible on={<#1->},
      edge={/tikz/visible on={<#1->}}},
    for all previous={
      for tree={
        alert on=\number\numexpr#1+1\relax,
        /tikz/visible on={<\number\numexpr#1+1\relax->},
        edge={/tikz/visible on={<\number\numexpr#1+1\relax->}}
      }},
    for children={edge={/tikz/visible on={<#1->}}}}}


\renewcommand{\firstrefdash}{}

% Color math mode!
\everymath{\color[rgb]{.19,.99,.95}}

% Highlight formal text (light blue)
\newcommand{\formal}[1]{\begin{color}[rgb]{.15,.4,.85}#1\end{color}}
% Highlight and bold new terminology (red)
\newcommand{\terminology}[1]{\alert{\textbf{#1}}}
% Translates as (black triple arrow)
\newcommand{\translates}[0]{$\color{black}\Rrightarrow$}
% Set brackets
\newcommand{\set}[1]{\{#1\}}
% Set brackets, shaded
\newcommand{\sett}[1]{\formal{\{#1\}}}
% Set brackets, math mode
\newcommand{\setm}[1]{$\{#1\}$}
% Ordered pair angled brackets
\newcommand{\oset}[1]{\langle #1\rangle}
% Semantic double brackets
\newcommand{\sem}[1]{\ensuremath{\llbracket #1 \rrbracket}}
% Cardinality bars
\newcommand{\card}[1]{\ensuremath{|#1|}}


%%STRIKETHROUGH MACRO
\def\str#1{{\setbox1=\hbox{#1}\leavevmode
      \raise.45ex\rlap{\leaders\hrule\hskip\wd1}
      \box1}}
%



\title{The Model of Syntax  }
\date{Autumn 2016}
\author{George Tsoulas}
\institute{Department of Language and Linguistic Science}
 \titlegraphic{\hfill\includegraphics[height=1cm]{../../../../graphics/logo}}

\begin{document}

\maketitle
\begin{frame}
  {The idea so far}

We discussed Xbars theory and Merge as the main tools for building and representing structure.

\end{frame}



\begin{frame}
  {Merge and Xbar}

  \begin{itemize}
\item    \texttt{Merge}:  is a procedural operation that tells us how to build the structure step by step.
  \item \texttt{X-bar} is a representational constraint that tells us essentially whether the output of merge is correct.
  \end{itemize}

\end{frame}


\begin{frame}

  We also saw that there was evidence for:

  \begin{itemize}
  \item Silent heads which would explain that even in the absence of an overt head the Xbar format is maintained.
  \end{itemize}

But we identified a problem:  Subjects

\end{frame}




\begin{frame}
  {Problem:  subjects}

Ok, so there are heads everywhere, but what about the other predictions of the theory, namely that every projection will have a specifier/subject?  

The following table shows that there appear to be serious differences between constituents in this respect.
\end{frame}

\begin{frame}

\begin{tabular}{l|lllllll}\hline
& C & T & D & P & A & V & N \\ \hline
Subjects & ? & DP & DP & ?&?&?&? \\ \hline
&&CP &&&&& \\ \hline
 & TP&VP&NP&DP&PP&DP&PP\\  
 &&&&PP&CP&PP&CP\\
Complements & &&&CP&&CP& \\
&&&&&&AP&\\
&&&&&&mult&\\\hline
&?&AdvP?&&AdvP&DegP&PP&PP\\
adjuncts &&&&&&AdvP&AP\\
&&&CP&&&&CP \\\hline
\end{tabular}


\end{frame}

\begin{frame}
  {And Yet....all is not lost}

Consider the following sentences:

\ex.
\a. Mary prefers \underline{her ice-cream in a cone}
\b. She considers \underline{John proud of his work}
\c. Henry found \underline{Bill sad}
\d.  They saw \underline{Bill leave}

The question here is what is the category of the underlined string?

Constituency tests show clearly that these are constituents.  So we can try some structures

\end{frame}



\begin{frame}
  \begin{center}
    \begin{forest}
      [VP [V [prefer]] [PP [DP [her ice cream, triangle]] [P' [in] [DP [a cone, triangle]]]]] 
    \end{forest}
  \end{center}
\end{frame}



\begin{frame}
  \begin{center}
    \begin{forest}
      [VP [V [consider]] [AP [DP [John, triangle]] [A' [proud] [PP [of his work, triangle]]]]] 
    \end{forest}
  \end{center}
\end{frame}


\begin{frame}
  \begin{center}
    \begin{forest}
      [VP [V [found]] [AP [DP [Bill, triangle]] [A' [sad]]]] 
    \end{forest}
  \end{center}
\end{frame}



\begin{frame}
  \begin{center}
    \begin{forest}
      [VP [V [saw]] [VP [DP [Bill, triangle]] [V [leave,]]]] 
    \end{forest}
  \end{center}
\end{frame}


\begin{frame}

So we should revise the top row of our table as follows:
\begin{center}
  \begin{tabular}{l|lllllll}\hline
         & C & T & D & P & A & V & N \\ \hline
Subjects & ? & DP & DP & \textcolor{red}{DP} & \textcolor{red}{DP}& \textcolor{red}{DP}&? \\ \hline
&&CP &&&&& \\ \hline
\end{tabular}
\end{center}
Which looks a lot better.

\end{frame}



\begin{frame}
  {Conclusions}
So we can conclude that, while a small amount of discrepancy between categories remains the overwhelming weight of evidence favours the approach that we have been developing so far. 

\end{frame}

\begin{frame}
  
\end{frame}


\begin{frame}
  {Thinking about Merge}

Merge does not build structure randomly.  In a way, it puts together only things that \textit{can} and \textit{should} be put together.  \pause

But \textit{how does Merge know what can and should be put together?} \pause

To answer this question we need to look into the information that is available to Merge, namely \textit{\textsl{\textbf{the lexical entry}}}.

\end{frame}

\begin{frame}
  {Lexical Entries}
We have already talked about lexical entries.    Let's complete the picture:
\begin{itemize}
\item Contents of a lexical entry, for an element LI (\textbf{L}exical \textbf{I}tem)
  \begin{itemize}
  \item The form of LI (is it pronounced, if yes how) pause
  \item The number of arguments it takes (from 0 to about 4) \pause
  \item The semantic relationship between the LI and its arguments. (More on this in a minute)
  \item The phrasal category that realises these arguments
  \item The syntactic configuration in which these arguments are realised (subject or complement)
  \item Other morpho-phonological properties
  \item Additional semantic information
  \end{itemize}
\end{itemize} 
\end{frame}

\begin{frame}
  {Semantic Relations} 

What we are concerned with here is the semantic relationships holding between a selecting head and its arguments.  These relations describe the status of the argument (or the individual denoted by the arguments) with respect to the action/event/state denoted by the verb.

\ex.
John ate an apple


\begin{itemize}
\item John is doing something $\rightarrow$ John is the \textsc{agent} of EAT.
\item The apple is affected by the action $\rightarrow$ the apple is the \textsc{theme}.
\end{itemize}

\end{frame}

\begin{frame}

\ex.
John loves Mary

Here John is not \textit{doing} anything, rather he experiences the relevant feeling. $\rightarrow$ John is the \textsc{experiencer}.  

\ex.
Bill baked cookies for Sue

Sue is the \textsc{beneficiary}.

and so on...

\end{frame}

\begin{frame}
  These relations are called \textit{Thematic} relations. (also $\theta$-relations).  In each such relations we have a thematic ($\theta$) assigner, which assigns a thematic ($\theta$) role to an argument.

The verb EAT assigns the $\theta$-role \textsc{theme} to its object and the $\theta$-role \textsc{agent} to its subject.
\end{frame}

\begin{frame}
  {$\theta$ Roles:  A list}

  \begin{itemize}
  \item Cause
  \item Agent
  \item Experiencer
  \item Location
  \item Goal
  \item Beneficiary
  \item Possessor
  \item Possessee
  \item Theme
  \end{itemize}

\end{frame}

\begin{frame}
  {The $\theta$-criterion}

\ex.
The $\theta$-Criterion\\
Every argument bears one and only one $\theta$-role

\end{frame}


\begin{frame}
  {Back to Lexical Entries}

With this in mind we can now give some examples of lexical entries:

\ex.
\textit{prefer} V  \underline{DP$_{exp}$} PP$_{theme}$

\ex.
\textit{Run} V  \underline{DP$_{agent}$}

\ex.
\textit{Eat} V  \underline{DP$_{agent}$} DP$_{theme}$


\end{frame}



\begin{frame}
  {C-selection and S-selection}

\begin{itemize}
\item We have already encountered the notion of \underline{C-selection}.  This corresponds to the specification in the lexical entry that tells us what sort of phrase a head selects.  \pause
\item But we just saw that we can look at selection in terms \textit{the role} that is being \textit{selected} (i.e. the verb EAT selects an agent and a theme).  This type of selection is called S(emantic)-selection. \pause
\item S-selection and C-selection are not completely independent from each other.
\end{itemize}
\end{frame}


\begin{frame}
  {More Lexical Entries}
  \begin{itemize}
  \item SEND:  V \underline{DP$_{agent}$} DP$_{theme}$ (P (to) DP)$_{Goal}$
  \item ELAPSE: V \underline{DP$_{theme}$}
  \item EXAMINE: V \underline{DP$_{agent}$} DP$_{theme}$
  \item THINK: V \underline{DP$_{agent}$} CP[that]$_{theme}$
  \item WONDER: V \underline{DP$_{agent}$} CP[+q]$_{theme}$
  \end{itemize}
\end{frame}



\begin{frame}
  {Adjectives}

  \begin{itemize}
  \item PROUD: A (\underline{DP$_{exp}$}) (PP$_{of}$)
  \item SAD: A  (\underline{DP$_{exp}$}) (PP$_{about}$)
  \end{itemize}
\end{frame}

\begin{frame}
  {Complementizers}
  \begin{itemize}
  \item THAT: C +tense TP[+tense]
  \item IF: C +tense, +q TP[+tense]
  \item FOR: C -tense TP[-tense]
  \item WHETHER C +q TP
  \end{itemize}
\end{frame}
\begin{frame}
  {Complementizers}
Simplifying:
  \begin{itemize}
  \item THAT: C +tense 
  \item IF: C +tense, +q
  \item FOR: C -tense
  \item WHETHER C +q
  \end{itemize}
\end{frame}

\begin{frame}
  {Tense etc}
  \begin{itemize}
  \item WILL: T[+tense]  DP$_{nom}$/CP VP, meaning: Future
  \item e: : T[+tense], bound, $\emptyset$, V,  DP$_{nom}$/CP VP, meaning: Present
\item ed: : T[+tense], bound, V,  DP$_{nom}$/CP VP, meaning: Past  
\item e: : T[-tense] VP, meaning: futurate
\end{itemize}


\end{frame}
\begin{frame}
  {Tense etc}
  \begin{itemize}
  \item WILL: T[+tense]  DP$_{nom}$/CP, meaning: Future
  \item e: : T[+tense], bound, $\emptyset$, V,  DP$_{nom}$/CP, meaning: Present
\item ed: : T[+tense], bound, V,  DP$_{nom}$/CP, meaning: Past  
\item e: : T[-tense], meaning: futurate
\end{itemize}
\end{frame}


\begin{frame}
  {The Projection Principle}

\begin{block}
{Definition}
Properties of Lexical Items must be satisfied.  Or:  Syntactic structure is projected from the Lexicon.
\end{block}

\end{frame}

\begin{frame}
  {Satisfying lexical properties}
Just like in morphology, lexical properties must be satisfied \textbf{Locally}.  Each head's requirements must be satisfied within its maximal projection. 
\end{frame}
\begin{frame}
  \begin{block}
    {Locality of Selection for Syntax}
    \begin{itemize}
    \item If $\alpha$ selects $\beta$ as complement, $\beta$ is a complement of $\alpha$.
    \item If $\alpha$ selects $\beta$ as subject, $\beta$ is the subject of $\alpha$. Or the subject of the \textit{clause containing $\alpha$.}
    \item If $\alpha$ selects $\beta$ as an adjunct, $\beta$ is an adjunct to $\alpha$.
    \end{itemize}
  \end{block}
\end{frame}

\begin{frame}
  {To Conclude:  The Model of Syntax}
  \begin{itemize}
  \item The atoms of Syntax are the outputs of the morphological component
  \item These Atoms can be assembled into syntactic complexes that can be represented in syntactic labeled trees
  \item The operation that combines these elements is called Merge, mapping a set of trees onto a new tree
  \item Merge is a recursive operation
  \item Well formed trees arise as a result of the following:
    \begin{itemize}
    \item Lexical properties of individual items determine the environments where they can occur
    \item The \textbf{Projection Principle} which states that properties of lexical items must be satisfied
    \item Locality of Selection:  Selecting elements are heads, selected elements are located within the maximal projection of the selecting head. (but see previous slide)
    \item n-ary branching:  Syntactic trees may have multiple branches.  In practice though, no more than 3 branches are found.
    \item Xbar theory which constraints the shape of syntactic objects. 
    \end{itemize}
  \end{itemize}

\end{frame}

\begin{frame}
  \begin{center}
    E\textsc{nd} O\textsc{f} P\textsc{art} I
  \end{center}
\end{frame}



Seminar Exercises:  Please prepare Exercises 2, and 5. 


\end{document}
