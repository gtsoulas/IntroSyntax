\documentclass[10pt]{beamer}
%{article}
%\usepackage{beamerarticle}
\usepackage{tipa}
\usetheme[
%sectiontitleformat=regular,
%everytitleformat=regular,
block=fill,
sectionpage=progressbar,
progressbar=foot,
numbering=fraction,
background=dark]{metropolis}
 
\setbeamercovered{invisible}

%\usecolortheme{owl}
\setbeamertemplate{blocks}[rounded][shadow=true]
\usepackage{booktabs}
\usepackage[scale=2]{ccicons}

\usepackage{pgfplots}
\usepgfplotslibrary{dateplot}


% Packages
\usepackage{linguexsf}
\usepackage{forest}
\usepackage{forest-animate}
\usetikzlibrary{arrows.meta} 
%\usepackage[usenames, dvipsnames]{color}
\usepackage{amsmath}
\usepackage{centernot}

  \tikzset{
    invisible/.style={opacity=0,text opacity=0},
    visible on/.style={alt=#1{}{invisible}},
    alt/.code args={<#1>#2#3}{\alt<#1>{\pgfkeysalso{#2}}{\pgfkeysalso{#3}}}}
\forestset{
  alert on/.style={
    /tikz/alt=<#1>{node options={/tikz/text=alert}}{}},
  dont show before/.style={
%    for first={alert on=#1},
    for ancestors'={
      /tikz/visible on={<#1->},
      edge={/tikz/visible on={<#1->}}},
    for all previous={
      for tree={
        alert on=\number\numexpr#1+1\relax,
        /tikz/visible on={<\number\numexpr#1+1\relax->},
        edge={/tikz/visible on={<\number\numexpr#1+1\relax->}}
      }},
    for children={edge={/tikz/visible on={<#1->}}}}}


\renewcommand{\firstrefdash}{}

% Color math mode!
\everymath{\color[rgb]{.19,.99,.95}}

% Highlight formal text (light blue)
\newcommand{\formal}[1]{\begin{color}[rgb]{.15,.4,.85}#1\end{color}}
% Highlight and bold new terminology (red)
\newcommand{\terminology}[1]{\alert{\textbf{#1}}}
% Translates as (black triple arrow)
\newcommand{\translates}[0]{$\color{black}\Rrightarrow$}
% Set brackets
\newcommand{\set}[1]{\{#1\}}
% Set brackets, shaded
\newcommand{\sett}[1]{\formal{\{#1\}}}
% Set brackets, math mode
\newcommand{\setm}[1]{$\{#1\}$}
% Ordered pair angled brackets
\newcommand{\oset}[1]{\langle #1\rangle}
% Semantic double brackets
\newcommand{\sem}[1]{\ensuremath{\llbracket #1 \rrbracket}}
% Cardinality bars
\newcommand{\card}[1]{\ensuremath{|#1|}}


%%STRIKETHROUGH MACRO
\def\str#1{{\setbox1=\hbox{#1}\leavevmode
      \raise.45ex\rlap{\leaders\hrule\hskip\wd1}
      \box1}}
%



\title{Introduction to Syntax}
\date{Autumn 2016}
\author{George Tsoulas}
\institute{Department of Language and Linguistic Science}
 \titlegraphic{\hfill\includegraphics[height=1cm]{../../../../graphics/logo}}

\begin{document}

\maketitle


\section{Introduction}
\begin{frame}[fragile]
  \begin{itemize}
  \item Linguistics is the \emph{scientific} study of language \pause
  \item The key term here is \emph{scientific}. Understanding how scientific study proceeds will help us in our study of syntax \pause
  \item Moreover, Linguistics is an \emph{empirical science}.  This means that there are part of the observable reality that we are trying to account for.
  \end{itemize}
\end{frame}


\begin{frame}[fragile]
\textcolor{yellow}{\textbf{\textit{To account for ?}}}  \ldots \pause  This means to provide the smallest and most general set of principles that describes the observed reality \textbf{and} allows us to predict what we would observe in similar or specifically different circumstances.  In other words \textit{\textbf{A Theory}}.
  
\end{frame}

\begin{frame}
  \frametitle{Theory of Language = Grammar}

The grammar of a language is a theory of that language,in the following sense:

\pause
A grammar is a set of rules.  These rules \textit{generate} in a specific sense (to which we will return) \textbf{All} and \textbf{Only} the \textbf{grammatical} sentences of the language.



\end{frame}


\begin{frame}
  There are basically two ways of reasoning that we can go about formulating a theory in any scientific field.
\end{frame}

\begin{frame}[fragile]
\frametitle{Reasoning methods}

\begin{center}
\begin{itemize}
\item  Inductive Reasoning 
\item Deductive reasoning
\end{itemize}
\end{center}
  
\end{frame}


\begin{frame}
  \frametitle{Inductive Reasoning}
Inductive Reasoning proceeds in the following steps:
\begin{enumerate}
\item Observation
\item Patterns
\item Tentative Hypothesis
\item Theory
\end{enumerate}
\end{frame}

\begin{frame}
  \frametitle{Deductive Reasoning}
  \begin{enumerate}
  \item Theory
  \item Hypothesis
  \item Observation
  \item Confirmation
  \end{enumerate}
\end{frame}


\begin{frame}
But whichever way we choose (and they are both useful), when we have nothing to start with we need to determine above all what is the reality that we are talking about and what observations will constitute the basic data.
\end{frame}

\begin{frame}[label={fr}]
  \frametitle{The Data}

The simplest form of data that a linguist can study is the speech stream that can be recorded, and the immediate observation that we can make is that almost any such recording will manifest:

\begin{itemize}
\item Sound
\item Words in a particular order 
\item Meaning
\end{itemize}

\pause


\begin{itemize}
\item[-] What is the nature of these three aspects of language ?
\item[-] How are they related ?
\end{itemize}

\end{frame}


\begin{frame}
  \frametitle{A caveat}

  \begin{block}
{Finding Data}
    The problem of determining what data is valuable and to the point is not an easy one. What is observed is often neither relevant nor significant, and what is relevant and significant is often very difficult to observe, in linguistics no less than \ldots anywhere in science.
\begin{flushright}
Noam Chomsky
\end{flushright}
  \end{block}

\end{frame}


\begin{frame}
  \frametitle{So, what \textit{is} data?}
  \begin{itemize}
  \item \textsc{Option I}
    \begin{itemize}
    \item Sentences that have been said or written in a language, say, English. \pause Imagine, for the sake of argument, that we can somehow collect them ... \textbf{all}. \pause
    \item[Problem 1:]  What if, despite the enormous quantity of sentences that we collect, there are aspects of the language that still end up not in the corpus?  perhaps a peculiar sentence type or something like that? \pause
    \item[Problem 2:]  The corpus will \textbf{by necessity} not contain sentences that are not used not because they are not grammatical but for some other reason (e.g. they are very long, or repetitive etc...) \pause
   \item[Problem 3:] For all its merits the corpus can tell you what has been done, not what \textit{can} be done.
   \item [Problem 4:] Can we be sure that this set of sentences will be constituted of grammatical sentences only?
     \end{itemize}
 \end{itemize}
\end{frame}


\begin{frame}
\frametitle{Option II}  

Let's take a step back; 
\begin{itemize}
\item[Question:] What do we really want to account for? \pause
\item[Answer:] \textbf{All} and \textbf{Only} the \textbf{grammatical} sentences of the language. \pause
\item[Question:] How can we tell whether a sentence is grammatical or not? \pause
\item[Answer:]If a native speaker of the language accepts the sentence as being part of their language then the sentence is grammatical. \pause
\item Clearly, however large the set of sentences that we can collect a speaker can always accept more.

\end{itemize}
\end{frame}

\begin{frame}
  \frametitle{Example}
  \begin{itemize}
  \item Jesse likes beans \pause
  \item Greg said that Jesse likes beans \pause
  \item Frank believes that Greg said that Jesse likes beans \pause
  \item Claire thought that Frank believes that Greg said that Jesse likes beans
  \item and so on and on....
  \end{itemize}
\end{frame}

\begin{frame}
  The point of this example is that we can go on forever like this and there is no guarantee that a collected corpus will contain all these sentences simply because they are too long and nobody says them. \emph{But} if asked to judge them \textit{everyone} will accept them as good sentences of English.
\end{frame}


\begin{frame}
  \frametitle{Example 2}

Any native speaker knows that there is a difference between the following sentences:

\ex.
Do you want tea or coffee ?

\ex.
* What do you want tea or ?

At a superficial level speakers know that \LLast is a perfectly \textit{grammatical} sentence of English, while \Last is not.
\end{frame}

\begin{frame}
  We can perform the same experiment with countless sentences and speakers will always be able to tell us, intuitively, which sentences are grammatical and which are not.  We can therefore formulate the following hypothesis:
  \begin{block}
    {Hypothesis}
There is something that speakers \textit{know} about their language and that knowledge underlies normal language use.  It is also this knowledge that we access when we need to judge a sentence as grammatical or not
  \end{block}
\pause
  \begin{block}
    {Hypothesis 2}
What we know about our language is \textit{The Grammar} of the language
  \end{block}
\end{frame}

\begin{frame}
\frametitle{Knowledge of Language}

\begin{itemize}
\item Speakers have knowledge of the principles and rules that govern their native language \pause
\item This knowledge is uniform (pathology aside) \pause
\item  This knowledge is tacit/implicit \pause
\item This knowledge is acquired early, effortlessly, and without need for any explicit instruction. \pause
\item Clearly this knowledge extends beyond syntax.  The same argument can be made for phonological properties, semantic properties and so on.
\end{itemize}
\end{frame}

\begin{frame}
  \frametitle{Conclusion 1: The object of study}

  \begin{block}
    {The Language Faculty}
There is a property of the mind/brain, a so-called \textit{epistemic state} of the mind/brain, which corresponds to the individual's knowledge of their language. 
  \end{block}
\pause
  \begin{block}
    {Universal Grammar}
The initial state (at birth) of the Language Faculty. Part of our genetic inheritance as humans.
  \end{block}
\pause
  \begin{block}
    {Grammar of Language $\mathcal{L}$}
The final state of the language faculty, in, say, adulthood.
  \end{block}

\end{frame}

\begin{frame}
\frametitle{Conclusion 2: The data of linguistics}

The fundamental data that we are concerned with is \textit{speakers' intuitions about specific sentences}, these are also known as \textit{Grammaticality judgements} \pause

This is not to say that there are no other sources of data.  There are and they are extremely valuable.  At the end of the day, a correct theory will be supported by various types of data.  

\end{frame}


\begin{frame}
  \frametitle{Grammaticality vs. Acceptability}

Whether a sentence is grammatical or not is determined by the rules of the grammar.  However (as we saw earlier) the grammar easily produces sentences that can be extremely long or very difficult to process.  The external factors that come into play (memory limitations for example) determine whether a sentence is \textit{acceptable} or not.  We will be concerned only with grammaticality judgements.
\end{frame}

\begin{frame}
\frametitle{Syntax}
 Syntax primarily deals with the second aspect of the data that we can observe,(see slide \ref{fr}), i.e.:
\begin{itemize}
\item Words in a particular order. (And from now on what we say even when it applies across domains should be taken as pertaining primarily to this)
\end{itemize}

Some of the questions that syntax raises are:
\end{frame}

\begin{frame}
  \frametitle{Syntactic Questions}

  \begin{itemize}
  \item Are all word orders permissible in any given language?
  \item Are those that are permissible equally meaningful?
  \item Are sentences just strings of words or more intricately structured constructs?
  \item How is structure (if any) built?
  \item What dictates which structures are available/permissible in any given language and which are not?
\item Are there syntactic properties that are universal?
\begin{itemize}
\item if yes, what are they?
\item if No, why not?
  \end{itemize}
\item  etc \dots
\end{itemize}
\end{frame}

\begin{frame}
  \frametitle{Structure}

Language is presented to us as hearers and produced by us as speakers in a linear fashion, i.e. one thing follows another and so on.

\pause
\begin{itemize}
\item Could this be all there is to the structure of sentences?
\pause
\item It could be, but it is not.
\end{itemize}

\end{frame}

\begin{frame}
  \frametitle{Levels of organisation}
Take a simple word like \textit{Speaker}, we have

\begin{itemize}
\item The phonemes: \textipa{'spi:k@}
\item The morphemes: \textsc{speak-}, \textit{er}
\item The word: Speaker
\end{itemize}

\end{frame}


\begin{frame}
We can represent the ways linguistic material is organised as follows:
\begin{center}
  \begin{forest}
    [SPEAKER
[SPEAK ['][s][p][i][:][k]] [ER [\textipa{@}]]]
  \end{forest}
\end{center}

\begin{itemize}
\item Phonemes on their own are not meaningful
\item Morphemes are the smallest unit that carries some meaning
\end{itemize}
\end{frame}

\begin{frame}
  The meaning of a morpheme can be of two types:

  \begin{itemize}
  \item Semantic 
  \item Grammatical
  \item or both...
  \end{itemize}

\end{frame}

\begin{frame}
\begin{itemize} 
\item  The semantic meaning of a morpheme tells us something about the denotation of the word.
\item The grammatical meaning tells us something about the grammatical category of the word (or something similar)
  \begin{itemize}
  \item Speak $\rightarrow$ Verb, denotes an event/action.
  \item -ER  $\rightarrow$ Nominalising morpheme (turns its host into a noun).
  \item Speak-er  $\rightarrow$ Noun, denotes an individual.
  \item Note that the morpheme ER on its own does not denote an individual.
  \end{itemize}
\end{itemize}
\end{frame}

\begin{frame}
What this example shows is that words, which we often take intuitively as basic are also structured in very precise ways and with very precise consequences.  

Consider the notion of a category.

\end{frame}

\begin{frame}
  \frametitle{Words and Categories}

Words come in categories like noun verb preposition etc \ldots.  There are two types of category:  \emph{Open Class} and \emph{Closed Class}  
\end{frame}

\begin{frame}
  \frametitle{Open Class Categories}

Open Class categories have a large number of members and, crucially, new elements can be freely added to them, synchronically.  The open class categories are:

\begin{itemize}
\item Noun
\item Verb
\item Adjective
\item Adverb
\end{itemize}


\end{frame}


\begin{frame}
  \frametitle{Closed Class categories}

Closed class categories have a small, finite number of elements and creating new elements in these categories is synchronically impossible.  The closed class categories are:

\begin{itemize}
\item Prepositions
\item Determiner
\item Numerals
\item Complementizers
\item Auxiliaries
\item Modals
\item Coordinators
\item Negation/Affirmation marker
\end{itemize}

\end{frame}

\begin{frame}
  \frametitle{Category changes}

Some morphemes change the category of an element.  This is only possible between open class category elements.

Note further that only the outer (or higher) category is relevant to any other processes.


\end{frame}

\begin{frame}
  \begin{center}
    \begin{forest}
      [N [V [SPEAK]] [N [ER]]]
    \end{forest}
  \end{center}
(We will see next week why this is always a noun and never a verb) Notice now that this can only take nominal inflections like Plural S but not verbal inflections like past tense or 3rd person S:

\begin{itemize}
\item Speakers
\item * Spoker
\item * Speakser 
\end{itemize}
\pause
How cool is that ????
\end{frame}
\begin{frame}
  \frametitle{To Conclude}

  \begin{itemize}
  \item Our object of study is people's knowledge of their language
  \item Grammaticality judgements allow us to probe that knowledge
  \item Syntax is concerned with structure
    \begin{itemize}
    \item Of sentences and phrases but also \pause
    \item of words
    \end{itemize}
  \end{itemize}
\end{frame}








\end{document}
