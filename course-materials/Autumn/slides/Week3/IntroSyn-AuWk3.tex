\documentclass[10pt]{beamer}
%{article}
%\usepackage{beamerarticle}
\usepackage{tipa}
\usetheme[
%sectiontitleformat=regular,
%everytitleformat=regular,
block=fill,
sectionpage=progressbar,
progressbar=foot,
numbering=fraction,
background=dark]{metropolis}
 
\setbeamercovered{invisible}

%\usecolortheme{owl}
\setbeamertemplate{blocks}[rounded][shadow=true]
\usepackage{booktabs}
\usepackage[scale=2]{ccicons}

\usepackage{pgfplots}
\usepgfplotslibrary{dateplot}


% Packages
\usepackage{linguexsf}
\usepackage{forest}
\usepackage{forest-animate}
\usetikzlibrary{arrows.meta} 
%\usepackage[usenames, dvipsnames]{color}
\usepackage{amsmath}
\usepackage{centernot}

  \tikzset{
    invisible/.style={opacity=0,text opacity=0},
    visible on/.style={alt=#1{}{invisible}},
    alt/.code args={<#1>#2#3}{\alt<#1>{\pgfkeysalso{#2}}{\pgfkeysalso{#3}}}}
\forestset{
  alert on/.style={
    /tikz/alt=<#1>{node options={/tikz/text=alert}}{}},
  dont show before/.style={
%    for first={alert on=#1},
    for ancestors'={
      /tikz/visible on={<#1->},
      edge={/tikz/visible on={<#1->}}},
    for all previous={
      for tree={
        alert on=\number\numexpr#1+1\relax,
        /tikz/visible on={<\number\numexpr#1+1\relax->},
        edge={/tikz/visible on={<\number\numexpr#1+1\relax->}}
      }},
    for children={edge={/tikz/visible on={<#1->}}}}}


\renewcommand{\firstrefdash}{}

% Color math mode!
\everymath{\color[rgb]{.19,.99,.95}}

% Highlight formal text (light blue)
\newcommand{\formal}[1]{\begin{color}[rgb]{.15,.4,.85}#1\end{color}}
% Highlight and bold new terminology (red)
\newcommand{\terminology}[1]{\alert{\textbf{#1}}}
% Translates as (black triple arrow)
\newcommand{\translates}[0]{$\color{black}\Rrightarrow$}
% Set brackets
\newcommand{\set}[1]{\{#1\}}
% Set brackets, shaded
\newcommand{\sett}[1]{\formal{\{#1\}}}
% Set brackets, math mode
\newcommand{\setm}[1]{$\{#1\}$}
% Ordered pair angled brackets
\newcommand{\oset}[1]{\langle #1\rangle}
% Semantic double brackets
\newcommand{\sem}[1]{\ensuremath{\llbracket #1 \rrbracket}}
% Cardinality bars
\newcommand{\card}[1]{\ensuremath{|#1|}}


%%STRIKETHROUGH MACRO
\def\str#1{{\setbox1=\hbox{#1}\leavevmode
      \raise.45ex\rlap{\leaders\hrule\hskip\wd1}
      \box1}}
%



\title{The Structure of Words:  Morphology }
\date{Autumn 2016}
\author{George Tsoulas}
\institute{Department of Language and Linguistic Science}
 \titlegraphic{\hfill\includegraphics[height=1cm]{../../../../graphics/logo}}

\begin{document}
\maketitle
\section{Words and Categories}
\begin{frame}
  \frametitle{Words and Categories}

We saw last week that words fall into categories (Noun, verb, determiner, adjective etc...).

\pause

But how do we know what category a word belongs to?


\end{frame}

\begin{frame}
\frametitle{Categorisation criteria}
  The main criterion that we use is the following:

  \begin{itemize}
  \item All words belonging to the same category behave in the same way with respect to certain properties.
  \end{itemize}
\end{frame}


\begin{frame}
  \frametitle{Distribution}

\begin{block}{Definition}
The \terminology{distribution} of a word or an element more generally is the sum of the different contexts where it can appear.  Other things being equal, all words belonging to the same category have the same distribution. 
\end{block}
\end{frame}



\begin{frame}
  \frametitle{Examples}

  \begin{itemize}
  \item Only nouns combine with the plural ending -s
  \item Only verbs combine with tense morphology or the progressive -ing
  \item An adjective with the element -ly at the end is an adverb
  \item Only adjectives can appear in the context: \textit{ The \underline{\hspace*{1cm}} Noun }
  \end{itemize}

\pause

All of the above have exceptions.


\end{frame}

\begin{frame}
  \frametitle{Nouns pluralise}

  \begin{itemize}
  \item Book $\rightarrow$ Book\textcolor{red}{\textbf{s}}
  \item Photo $\rightarrow$ Photo\textcolor{red}{\textbf{s}}
  \item Basket $\rightarrow$ Basket\textcolor{red}{\textbf{s}}
  \item Pillow $\rightarrow$ Pillow\textcolor{red}{\textbf{s}}
  \item Organisation $\rightarrow$ Organisation\textcolor{red}{\textbf{s}}

  \end{itemize}

Compare to:

\end{frame}

\begin{frame}
\frametitle{Pluralisation Cont'd}

  \begin{itemize}
  \item They eat $\rightarrow$ *They eat\textcolor{red}{\textbf{s}}
  \item Yellow $\rightarrow$ *Yellow\textcolor{red}{\textbf{s}}
  \item Passionately $\rightarrow$ *Passionately\textcolor{red}{\textbf{s}}
  \item Under $\rightarrow$ *Under\textcolor{red}{\textbf{s}}
  \item And $\rightarrow$ *And\textcolor{red}{\textbf{s}}

  \end{itemize}

\end{frame}

\begin{frame}
\frametitle{But}

\begin{itemize}
\item *Furniture\textcolor{red}{s}
\item No if\textcolor{red}{s} no but\textcolor{red}{s}
\item I take the red\textcolor{red}{s} you take the blue\textcolor{red}{s}
\end{itemize}

  
If we look closely we see that these are special cases, which, if anything reinforce our conclusions


\end{frame}


\begin{frame}
 \frametitle{Verbs combine with Tense}

  \begin{itemize}
  \item I look $\rightarrow$ I look\textcolor{red}{\textbf{ed}}
  \item Sue listens $\rightarrow$ Sue listen\textcolor{red}{\textbf{ed}}
  \item Jesse smokes $\rightarrow$ Jesse smok\textcolor{red}{\textbf{ed}}
  \item Joe rides a bike $\rightarrow$ Joe r\textcolor{red}{\textbf{ode}} a bike
  \end{itemize}

\end{frame}

\begin{frame}
\frametitle{Tense cont'd}
  Comparing with other categories gives:


  \begin{itemize}
  \item Organisation $\rightarrow$ *Organisation\textcolor{red}{\textbf{ed}}
  \item Quickly $\rightarrow$ *Quickly\textcolor{red}{\textbf{ed}}
  \item Metallic $\rightarrow$ *Metallic\textcolor{red}{\textbf{ed}}
  \item At $\rightarrow$ At\textcolor{red}{\textbf{ed}}
  \end{itemize}

\end{frame}


\begin{frame}
  \frametitle{Adjectives}


The $ \left[ \begin{array}{l}Blue\\Old\\interesting\\fat \end{array} \right] $ Book \emph{vs.} The $ \left[ \begin{array}{l}*of\\ *swiftly\\ *listen \\ pencil  \end{array} \right]$ Book

\bigskip

The case of:


\ex.  The pencil book


is one that we need to think about a little more.

\end{frame}


\begin{frame}
\frametitle{Complementary Distribution}

Sometimes the fact that two elements cannot appear at the same time shows us that they are of the same category and they compete for the same structural position in the sentence:

\ex.
\a. The book
\b. John's book
\c. *The John's book
\d. *John's the book

\end{frame}

\section{Morphemes and Words}
\begin{frame}[fragile]
\frametitle{Morphemes and Words}

\begin{itemize}
\item Morphemes are the smallest units that carry (some kind of) meaning
\item Words are made up of morphemes.
\end{itemize}

\end{frame}


\begin{frame}
\frametitle{Morphemes in a word}
  Some words are made of a single morpheme.  They are called \textit{mono-morphemic}

\ex.
\a. Pen
\b. Tea
\c. Book
\d. Ink

Not only short words are mono-morphemic

\ex.
\a. Window
\b. Elephant
\c. Encyclopedia
\d. Paragraph
\e. Linguistics

\end{frame}

\begin{frame}
\frametitle{Poly-morphemic words}
Other words contain more than one morpheme.  

\ex.
\a. Constitute
\b. Constitu-tion
\c. Constitu-tion-al
\d. Constitu-tion-al-ity
\d. Constitu-tion-al-itie-s
\e. Anti-constitu-tion-al-it-ies


These are called \textit{poly-morphemic}.  If a word contains just two morphemes it is called \textit{bi-morphemic}

\end{frame}

\subsection{Types of Morphemes}
\begin{frame}
  \frametitle{Types of Morphemes}

Every word that contains more than one morpheme has a core part usually called \textit{the stem} (sometimes it is also referred to as \textit{the root}).  This is the part to which morphemes attach and which determines the core meaning of the word.  In the previous example, repeated here:

\ex.
\a. Constitut-e
\b. Constitu-tion
\c. Constitu-tion-al
\d. Constitu-tion-al-ity
\d. Constitu-tion-al-itie-s
\e. Anti-constitu-tion-al-it-ies
\f. Constitu-tion-al-ly

\textit{Constitut-} is the stem.  Every other of the words in the example mean something different in virtue of their grammatical category but they are related to the meaning of the stem of \textit{constitut-}

(But you can already see that the application of this idea is tricky)

\end{frame}

\begin{frame}
\frametitle{Types of Morphemes cont'd}

We can classify morphemes in a variety of ways:

\begin{itemize}
\item By category/function
  \begin{itemize}
  \item Nominalising
  \item Verbalising
  \item Adjectival
  \item etc....
  \end{itemize}

\end{itemize}
\end{frame}


\begin{frame}
  \frametitle{Types of Morphemes cont'd}

We can also classify them with respect to the position where they attach with respect to the stem.

\begin{itemize}
\item At the beginning of the stem $\rightarrow$ \terminology{Prefix}
  \begin{itemize}
  \item Pre-, Anti-, Post-, Re-, Dis-, Un-, Over- 
  \item and so on 
  \end{itemize}
\item At the end of the stem $\rightarrow$ \terminology{Suffix}
  \begin{itemize}
  \item -able, -dom, -er, -ism, -ship, -en, -ify,-ful, -ical, -less
  \item and many others
  \end{itemize}
\item In the middle of the stem $\rightarrow$ \terminology{Infix}
  \begin{itemize}
  \item English does not have real infixes but \textit{Fucking} is one example as in \textit{Fan-fucking-tastic}
  \end{itemize}

\end{itemize}

\end{frame}
\frametitle{Testing Infixes}
To see that \textit{Fucking} is an infix in this particular use consider the fact that there is a difference between \Next and \NNext :
\ex. Fucking Fantastic

\ex. Fan-Fucking-tastic

Even better:

\ex. 
\a. San fran-fucking-cisco 
\b. * San fucking Francisco

We can therefore conclude that \textit{Fucking}, in these cases is really used as an infix.

\begin {frame}
\frametitle{More morpheme types}
  \begin{itemize}
  \item At the beginning \textit{and the end} of the stem $\rightarrow$ \terminology{Circumfix}.
  \end{itemize}

Examples of circumfixes in English are not easy to come by.  One possible case would be the dialectal formation of the progressive (the circumfix is in red):

\ex.
We will no more go \textcolor{red}{\textbf{a}}rov\textcolor{red}{\textbf{ing}}

In other languages they are more common, the formation of the perfect tenses in Ancient Greek for example involves two parts:

\begin{itemize}
\item Reduplication of the initial consonant followed by the vowel E
\item The morpheme -k at the end of the stem.
\end{itemize}


\ex. \textcolor{red}{\textbf{le}}-lu-\textcolor{red}{\textbf{k}}a = I have loosened

\end{frame}

\begin{frame}
\frametitle{Affix, Bound, Free Morpheme}
  The general term for all the  \textit{fixes} (suf-, pre-, in-, circum-) is 
  \begin{center}
    \terminology{AFFIX}
  \end{center}
\pause

Affixes are also called \textit{bound} morphemes, in opposition to \textit{free} morphemes because they must attach to a host and cannot be pronounced separately.  \textit{Bound} vs. \textit{Free} is a phonological property.


\end{frame}
\begin{frame}
  \frametitle{Finding morphemes}

How do we know that a particular element is a morpheme?

Essentially, we apply the same method as we saw for words, namely, we look at their distribution. 

\end{frame}


\begin{frame}
\frametitle{Finding Morphemes cont'd}
  We need to be careful, however, to list only the morphemes that are really active in the language at a given time.  Take:

\ex.  Parachute

Although there is in modern English a word \textit{chute} knowing its meaning does not help much in figuring out what the meaning of \textit{parachute} is.  If you know French, on the other hand, you know that 

\ex.
Chute = Fall

If you know ancient Greek you also know that one of the meanings of the preposition \textit{Para} is \textit{against}.  So you figure out that:

\ex.
Parachute = against the fall


\end{frame}
\begin{frame}
\frametitle{Finding Morphemes cont'd}
  But is this part of the relevant knowledge of English that a normal English speaker has and we want to provide a theory for?
  Clearly not.  We saw that with \textit{Chute}.  The same is true of \textit{Para}, people do not interpret:

\ex.
\a. Parakeet $\centernot\rightarrow$ Against the keets
\b. Paracetamol $\centernot\rightarrow$ Against the cetamols

The point here is that the fact that a particular element is derived from a borrowed word or has a particular historical source, should not be taken as a criterion unless we can be sure that speakers have the relevant knowledge.
 
\end{frame}

\section{Order and Selection} 

\begin{frame}
  \frametitle{Morpheme Order}

\begin{itemize}
\item Morphemes combine is specified orders which are strictly limited.  The reason for this is that morphemes (as we saw), have particular grammatical roles.  The morphemes that we are looking at serve, mainly to change the category of the thing they attach to. \pause
\item Morphemes are selective.  They do not attach to \emph{any} word and change its category:
  \begin{itemize}
  \item Nation + al = Noun + al = National = Adjective
  \item Quickly + al = adverb + al = *Quicklyal = nothing 
  \end{itemize}
\end{itemize}
How do we interpret this ?

\end{frame}

\begin{frame}
\frametitle{C-selection}

Morphemes \terminology{select} the element (which we will call \terminology{constituent} without defining this term yet) that they attach to.  This is called \terminology{C-selection}.

The C-selectional properties of morphemes are specified in their \terminology{lexical entries}
  
\end{frame}

\begin{frame}
  \frametitle{Lexical Entries}

A lexical entry is a memory address where we keep information about each element of the lexicon.  Elements of the lexicon are called \terminology{Lexical Items}.  Each lexical item has its own lexical entry.  The lexical entry contains phonological, syntactic and semantic information.  For the morpheme \textit{-al} the portion of its lexical entry related to C-selection will state:
\pause
\begin{center}

  \begin{tabular}[t]{|l|l|} \hline
  -al  & C-selects: N\\ \cline{2-2}
       & Creates: Adjective (N+al=Adj)\\ \hline
  \end{tabular}
\end{center}
 
And so on for other morphemes


\end{frame}

\begin{frame}
\frametitle{Example} 
 A poly-morphemic word illustrates clearly the process, which is recursive:

\begin{center}
  \begin{forest}
    [N   [V [A [ N [Nation]] [-al]] [-ize]] [-ation]]
  \end{forest}
\end{center}

\end{frame}


\begin{frame}
  \frametitle{Tree Diagrams}
Structure is represented in tree diagrams with the root at the top:

\begin{center}
  \begin{forest}
    [$\alpha$ [$\beta$] [$\gamma$]]
  \end{forest}
\end{center}

\begin{itemize}
\item The basic relation in a tree structure is that of \terminology{Dominance} which is represented in top-to-bottom order.  A node $\alpha$ dominates the nodes that are below it.
\item  \terminology{Immediate Dominance}: A node $\alpha$ immediately dominates a node  $\beta$ if and only if:
  \begin{itemize}
  \item  $\alpha$ dominates  $\beta$ and 
  \item There is no node $\gamma$ distinct from  $\alpha$ such that   $\gamma$ dominates $\beta$ and does not dominate   $\alpha$
  \end{itemize}
\end{itemize}
\end{frame}




\begin{frame}
  \frametitle{Tree terminology}

  \begin{itemize}
  \item The top node is called \terminology{the root}.
  \item The lines connecting nodes are called \terminology{Branches}.
  \item A branching node is a \terminology{Mother node}.
  \item The nodes immediately dominated by a given node are its \terminology{Daughters}.
  \item Two nodes immediately dominated by the same node are \terminology{sisters}.
\item A node that does not dominate anything is called a \terminology{leaf} or a \terminology{terminal node}.
  \end{itemize}
\end{frame}


\begin{frame}
  \frametitle{Locality of Selection}

  \begin{block}
    {Definition}
Selection is \emph{local}. Any given element may only select properties of its sister.
  \end{block}

NB.  Make sure you convince yourselves of this by applying it to as many cases as you can think.


\end{frame}

\begin{frame}
  \frametitle{Compounds}


  \begin{block}
    {Definition}
A compound is the word that we obtain when we combine two otherwise independent words.  An independent word is a free morpheme.  
  \end{block}


\end{frame}


\begin{frame}
  \frametitle{Compounds in English}

  \begin{itemize}
  \item[Question:] If I combine two free morphemes X and Y is the result an X or a Y?
  \item[Answer:] In English the category of a compound is determined by the element on the right (the one in red in the tree)  
  \end{itemize}

  \begin{center}
    \begin{forest}
      [\textcolor{red}{Y}[X][\textcolor{red}{Y}]]
    \end{forest}
  \end{center} \pause
  \begin{block}
    {The English Right-Hand Head Rule}
The rightmost element of the compound is the head of the compound.
  \end{block}
\end{frame}



\begin{frame}
  It is easy to verify that this generalisation is correct:


  \begin{itemize}
  \item Handbag
  \item Butterfly
  \item Pushchair
  \item Plant-pot
  \item Armrest
    \item etc...
  \end{itemize}


\end{frame}


\begin{frame}
\frametitle{Consequences of the RHHR}
  A further consequence of the RHHR is that the rightmost element also determines other grammatical properties such as number and (in the languages that have it gender)

\ex.  
\a. Towel racks: plural
\b. Part suppliers: plural
\c. Parts supplier: singular


\end{frame}


\begin{frame}
\frametitle{Generalising}

Although the RHHR was seen in action with compounds we can ask whether it is also valid for all cases.  Take again \textit{nationalisation}

\begin{center}
  \begin{forest}
    [N   [V [A [ N [Nation]] [-al]] [-ize]] [-ation]]
  \end{forest}
\end{center}

And think of each process of affixation.
\end{frame}

\begin{frame}
\frametitle{Examples}
\begin{center}
\begin{tabular}{lll}
\begin{forest}
[\textcolor{red}{A} [N [Nation]] [\textcolor{red}{A} [-al]]]
\end{forest}
&
\begin{forest}
   [\textcolor{red}{V} [A [national]] [\textcolor{red}{V} [-ize]]]
\end{forest}
&
\begin{forest}
  [\textcolor{red}{N} [V [nationalize]] [\textcolor{red}{N} [-ation]]]
\end{forest}
\end{tabular}
\end{center}

It is clearly the element on the right that does the job.

\end{frame}


\begin{frame}
\frametitle{The RHHR: the general version}
  \begin{block}
    {The English Right-Hand Head Rule}
The rightmost element of a word is the head of the word.
  \end{block}
  
\end{frame}


\begin{frame}
  \frametitle{A Consequence}

The RHHR will apply to all morphological combinations, i.e. the category of the whole will be determined by the category of its rightmost element, this means that:
\bigskip

\begin{tabular}[c]{lll}
  \begin{forest}
    [\textcolor{red}{T}[ V [smok-]][\textcolor{red}{T} [ed]]] 
  \end{forest}
  &
   \begin{forest}
    [\textcolor{red}{Num}[ V [Book-]][\textcolor{red}{Num} [s]]] 
  \end{forest}
  &
etc \dots
\end{tabular}

Morphemes like \textit{-ed} and \textit{-s} are called \terminology{Inflectional}

\end{frame}

\begin{frame}

\frametitle{A problem for the RHHR}

How about derived categories when there is no morpheme that can be associated with the new category?  Consider:


\ex. Father $\neq$ The one who Faths

Father is a noun and the derived verb is also \textit{father}

  
\ex.  
Max has fathered three children

\end{frame}



\begin{frame}
\frametitle{A Solution}
  For these cases we postulate that the process is exactly the same but the morpheme that turns the noun into a verb phonetically null (has no pronunciation):

\begin{tabular}[t]{lllll}
  \begin{forest}
    [\textcolor{red}{V}[ N [father-]][\textcolor{red}{V} [$\emptyset$]]] 
  \end{forest}
  &
&
   \begin{forest}
    [\textcolor{red}{T}[ V [laugh-]][\textcolor{red}{T} [ed]]] 
  \end{forest}
  &
 &
  \begin{forest}
    [\textcolor{red}{T}[ V [N [father]] [V [$\emptyset$]]][\textcolor{red}{T} [ed]]] 
  \end{forest}

\end{tabular}

\end{frame}

\begin{frame}
  \frametitle{Silent Affixes}

This approach \textit{predicts} that we will not be able to attach to elements that already have a silent affix another affix that C-selects for a root.  This is correct.

\ex.
\a. * Father-age
\b. * Father-al
\c.  * Father-ant
\d. * Father-ance
\e. Etc\dots


\end{frame}

\begin{frame}
\frametitle{Silent affixes and Meaning}

Silent affixes have as much meaning as their overt cousins.

\ex. 
Wet $\rightarrow$ ``to make wet''


This means that the meaning of the silent affix is that of causation (equivalent to \textit{make})

\end{frame}

\begin{frame}
  \frametitle{Compositionality}

  \begin{block}
    {Compositionality in Morphology}
The morphological, syntactic and semantic properties of (at least some) complex words are determined by their parts and how those parts are put together.
  \end{block}
\end{frame}


\begin{frame}
  \begin{block}
    {Recursion in Morphology}
Morphology can have recursive affixation and recursive compounding.  When this happens, the language has infinitely many words.   
\end{block}
\end{frame}

\section{Conclusion}
\begin{frame}
  \frametitle{Conclusion}
  \begin{itemize}
  \item Words have structure
  \item Words are made up of smaller elements the morphemes
  \item Word structure is compositional and recursive
  \item Word Structure is represented in Tree diagrams
  \end{itemize}






\end{frame}


\end{document}
