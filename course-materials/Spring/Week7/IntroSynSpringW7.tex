\documentclass[10pt]{beamer}
%{article}
%\usepackage{beamerarticle}
\usepackage{tipa}
\usetheme[
%sectiontitleformat=regular,
%everytitleformat=regular,
block=fill,
sectionpage=progressbar,
progressbar=foot,
numbering=fraction,
background=dark]{metropolis}
 
\setbeamercovered{invisible}

%\usecolortheme{owl}
\setbeamertemplate{blocks}[rounded][shadow=true]
\usepackage{booktabs}
\usepackage[scale=2]{ccicons}

\usepackage{pgfplots}
\usepgfplotslibrary{dateplot}


% Packages
\usepackage{linguexsf}
\usepackage{forest}
\usepackage{forest-animate}
\usetikzlibrary{arrows.meta} 
%\usepackage[usenames, dvipsnames]{color}
\usepackage{amsmath}
\usepackage{centernot}

  \tikzset{
    invisible/.style={opacity=0,text opacity=0},
    visible on/.style={alt=#1{}{invisible}},
    alt/.code args={<#1>#2#3}{\alt<#1>{\pgfkeysalso{#2}}{\pgfkeysalso{#3}}}}
\forestset{
  alert on/.style={
    /tikz/alt=<#1>{node options={/tikz/text=alert}}{}},
  dont show before/.style={
%    for first={alert on=#1},
    for ancestors'={
      /tikz/visible on={<#1->},
      edge={/tikz/visible on={<#1->}}},
    for all previous={
      for tree={
        alert on=\number\numexpr#1+1\relax,
        /tikz/visible on={<\number\numexpr#1+1\relax->},
        edge={/tikz/visible on={<\number\numexpr#1+1\relax->}}
      }},
    for children={edge={/tikz/visible on={<#1->}}}}}


\renewcommand{\firstrefdash}{}

% Color math mode!
\everymath{\color[rgb]{.19,.99,.95}}

% Highlight formal text (light blue)
\newcommand{\formal}[1]{\begin{color}[rgb]{.15,.4,.85}#1\end{color}}
% Highlight and bold new terminology (red)
\newcommand{\terminology}[1]{\alert{\textbf{#1}}}
% Translates as (black triple arrow)
\newcommand{\translates}[0]{$\color{black}\Rrightarrow$}
% Set brackets
\newcommand{\set}[1]{\{#1\}}
% Set brackets, shaded
\newcommand{\sett}[1]{\formal{\{#1\}}}
% Set brackets, math mode
\newcommand{\setm}[1]{$\{#1\}$}
% Ordered pair angled brackets
\newcommand{\oset}[1]{\langle #1\rangle}
% Semantic double brackets
\newcommand{\sem}[1]{\ensuremath{\llbracket #1 \rrbracket}}
% Cardinality bars
\newcommand{\card}[1]{\ensuremath{|#1|}}


%%STRIKETHROUGH MACRO
\def\str#1{{\setbox1=\hbox{#1}\leavevmode
      \raise.45ex\rlap{\leaders\hrule\hskip\wd1}
      \box1}}
%



\title{More Movement}
\date{Spring 2017}
\author{George Tsoulas}
\institute{Department of Language and Linguistic Science}
 \titlegraphic{\hfill\includegraphics[height=1cm]{../../../../graphics/logo}}

\begin{document}
\maketitle

 \begin{frame}
   {Recap on the displacement property}

   \begin{itemize}
   \item Elements of a structure can appear in positions that do not correspond to the positions where they are selected by a selecting head.
\item This is what we call the \textit{displacement property} of language.
\item The operation that creates the structures where this property is apparent is \textit{Movement}.
\item Movement can be seen as an operation that maps one structure (Tree), to another (Tree).
\item The reason we call it movement is because it is a though the operation moved an element from one spot to another.  
   \end{itemize}
 \end{frame}



 \begin{frame}
   {Diagnosing Movement}
The general idea:

How do we know that $\alpha$ has moved from position $\beta$ to position $\gamma$?  \pause In other words: \pause

What evidence do we have (and can come up with) to convince ourselves that $\alpha$ has been first to $\beta$ before reaching $\gamma$? \pause

The evidence for movement is somewhat different for each type of movement. 


 \end{frame}



 \begin{frame}
{Types of Movement}

So far we have claimed (correctly) that only two things can move, namely:

\pause

\begin{itemize}
\item Heads
\item Phrases
\end{itemize}

In week 5 we talked mostly of movement of heads.
   
\end{frame}

\begin{frame}
  {Head Movement again}


The main diagnostic for the movement of heads follows from the idea (that we defended in the Autumn term) that bound morphemes are heads H that project their own HP.  What follows immediately from this is this:

\begin{enumerate}
  \item That their bound character has to be derived in some way given that in the syntactic structure there could be something that intervenes between the bound morpheme and the stem/root that it is \textit{bound} to.
\end{enumerate}

\end{frame}

\begin{frame}
  Another thing that follows is that we can use the test of intervention by other elements to determine the type of head-movement that has taken place (Raising vs. Lowering)
\end{frame}

\begin{frame}
  {Other diagnostics}

Detecting movement amounts to detecting selection.  In other words if there is evidence that an element is \textit{selected} at a position which is different from the one it appears in then we can be confident that we have a case of movement.

\end{frame}

\begin{frame}
  {Four ways of detecting selection}
  \begin{enumerate}
  \item Covariation
  \item Idiom chunks and weather it (Weather \textit{it} will be left aside until chapter 9)
  \item Case
  \item Existential Constructions
  \end{enumerate}
\end{frame}

\begin{frame}
  {Co-variation}

Look at what covaries with what:

\begin{enumerate}
\item Selecting a verb
\item Asking what entities the verb relates 
\item finding what strings of morphemes correspond to these entities
\end{enumerate}

\end{frame}

\begin{frame}
  \ex.
Time seems to elapse slowly in the Tropics

\ex.
* Mary seems to elapse slowly in the tropics


\ex.
Mary seems to speak slowly in the Tropics

\end{frame}



\begin{frame}
  {Idiom chunks and weather \textit{it}}

\ex.
\a. Pull strings
\b. Take care of
\c. Make headway
\d. Lend assistance

These elements have a very tight relationship, the idiomatic meaning can only appear in certain configurations.


\end{frame}

\begin{frame}
  The following are strange:

\ex.
Strings are the things that you can pull when you want something done

\ex. 
Care is the thing that you take of/from your relatives


\ex.
Headway is the thing that you make when you try 


\ex.  etc....


\end{frame}

\begin{frame}
  What the strangeness of the above examples shows is that the tight relationship between the elements of the idiom chunk cannot be broken in this way.  However, the following are fine:
\end{frame}


\begin{frame}
  \ex.
How many strings did you say she had to pull in order to do that?

\ex.
How much care do you think he would be taking of his patients under those circumstances?

\ex.
How much headway is he likely to make?


\end{frame}

\begin{frame}
  The fact that the closely selected noun dos not appear next to the selecting verb and yet the interpretation is that of the idiom chunk implies that movement has taken place.


  \begin{block}
    {Exercise}
Go back to the examples that I suggested were strange and explain the way in which they are different from those that are OK
  \end{block}
\end{frame}


\begin{frame}
{Case}

DPs receive case depending on the position they occur.  In the case of pronouns where case is overtly realised if the pronoun appears in a position distinct from its case position, then movement has taken place:

\ex. 
\a. Who left Bill
\b. *Whom left Bill


\ex.
\a. Who did Bill leave
\b. Whom did Bill leave


  
\end{frame}


\begin{frame}
  {Phrasal Movements}


Apart from heads, phrases can also move.  There are several types of phrasal movement and we will return to them in more detail but we can look at two types for now.

\end{frame}


\begin{frame}
  {WH-Movement}

This is the type of movement that is seen in so-called WH-questions (which should be differentiated from Yes/No questions).

\begin{enumerate}
\item A WH-question is a request for information on a particular element of the sentence
\end{enumerate}

\ex.
What did Mary eat?

Now if you think of the possible answers to this question you realise the simple fact that the answers are as follows:


\end{frame}


\begin{frame}

  \ex. Mary ate \textit{a chicken}

\ex. Mary ate \textit{a cookie}

\ex.  Mary ate \textit{Kale}


\end{frame}


\begin{frame}
  The conclusion from these simple examples is that in the simple question 

\ex. 
What did Mary eat


the WH-word \textit{what} appears away from the position in which it was selected (which was the position after the verb).


\end{frame}

\begin{frame}
  So, thinking of the original formulation of movement as an operation that maps one structure to another we can say that the one of the two sentences so related is:

\ex.
Mary did eat what

Note that this is not a well formed sentence.


\end{frame}


\begin{frame}
  After head movement of \textit{did} to C (T-to-C):

\ex.
Did Mary eat what


Which is again not a well formed sentence.


\end{frame}

\begin{frame}
  Finally after WH-movement of what to the front (to spec CP):

\ex.
What did Mary eat


Which is a well formed sentence of English (a question)

\end{frame}

\begin{frame}
  {Raising to subject}

Another type of movement is what we see in sentences like the one we saw earlier:

\ex.
Time seems to elapse slowly in the Tropics

The DP \textit{Time} is not selected by the verb \textit{seem}.


\end{frame}


\begin{frame}
  We can see this is by looking at the following pair:

\ex.
John seems to appreciate/smell/like cheese


\ex.
John seems to punch the bag


The $\theta$ role assigned by appreciate/smell/like is \textit{experiencer}, i.e. the individual who experiences a particular sensation that is not caused voluntarily by some agent.

The $\theta$ role assigned by \textit{punch} is \textit{agent}.


\end{frame}

\begin{frame}
  Furthermore the subject position of \textit{Seem} can be filled by the expletive element \textit{It}:

\ex.
It seems (that) John likes cheese


We cannot say that \textit{It} is \textit{selected} by seem.

\end{frame}


\begin{frame}
  The point is that seem does not select and does not $\theta$ mark its subject.  As a result, the matrix subject position is available for the embedded subject to move into. 

(The precise motivations will be the topic of chapter 9)

\end{frame}

\begin{frame}
  So in the same vein as earlier, we can say that we start with:

\ex.
seems John to like cheese

which is not a well-formed sentence.


\end{frame}

\begin{frame}
  After raising of the embedded subject to the matrix subject position (matrix Spec TP)

\ex.
John seems to like cheese

which is a well formed sentence


\end{frame}

\begin{frame}
  {Conclusion}

  \begin{enumerate}
  \item There is ample evidence that there are movement processes in language.
  \item Movement involves heads and phrases
  \item Diagnosing movement = diagnosing selection
  \item Two of the possible phrasal movements are:
    \begin{enumerate}
    \item WH-movement which forms questions (and in fact much more)
    \item Raising to subject
    \end{enumerate}

  \end{enumerate}
\end{frame}




\end{document}
