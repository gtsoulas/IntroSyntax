\documentclass[10pt]{beamer}
%{article}
%\usepackage{beamerarticle}
\usepackage{tipa}
\usetheme[
%sectiontitleformat=regular,
%everytitleformat=regular,
block=fill,
sectionpage=progressbar,
progressbar=foot,
numbering=fraction,
background=dark]{metropolis}
 
\setbeamercovered{invisible}

%\usecolortheme{owl}
\setbeamertemplate{blocks}[rounded][shadow=true]
\usepackage{booktabs}
\usepackage[scale=2]{ccicons}

\usepackage{pgfplots}
\usepgfplotslibrary{dateplot}


% Packages
\usepackage{linguexsf}
\usepackage{forest}
\usepackage{forest-animate}
\usetikzlibrary{arrows.meta} 
%\usepackage[usenames, dvipsnames]{color}
\usepackage{amsmath}
\usepackage{centernot}

  \tikzset{
    invisible/.style={opacity=0,text opacity=0},
    visible on/.style={alt=#1{}{invisible}},
    alt/.code args={<#1>#2#3}{\alt<#1>{\pgfkeysalso{#2}}{\pgfkeysalso{#3}}}}
\forestset{
  alert on/.style={
    /tikz/alt=<#1>{node options={/tikz/text=alert}}{}},
  dont show before/.style={
%    for first={alert on=#1},
    for ancestors'={
      /tikz/visible on={<#1->},
      edge={/tikz/visible on={<#1->}}},
    for all previous={
      for tree={
        alert on=\number\numexpr#1+1\relax,
        /tikz/visible on={<\number\numexpr#1+1\relax->},
        edge={/tikz/visible on={<\number\numexpr#1+1\relax->}}
      }},
    for children={edge={/tikz/visible on={<#1->}}}}}


\renewcommand{\firstrefdash}{}

% Color math mode!
\everymath{\color[rgb]{.19,.99,.95}}

% Highlight formal text (light blue)
\newcommand{\formal}[1]{\begin{color}[rgb]{.15,.4,.85}#1\end{color}}
% Highlight and bold new terminology (red)
\newcommand{\terminology}[1]{\alert{\textbf{#1}}}
% Translates as (black triple arrow)
\newcommand{\translates}[0]{$\color{black}\Rrightarrow$}
% Set brackets
\newcommand{\set}[1]{\{#1\}}
% Set brackets, shaded
\newcommand{\sett}[1]{\formal{\{#1\}}}
% Set brackets, math mode
\newcommand{\setm}[1]{$\{#1\}$}
% Ordered pair angled brackets
\newcommand{\oset}[1]{\langle #1\rangle}
% Semantic double brackets
\newcommand{\sem}[1]{\ensuremath{\llbracket #1 \rrbracket}}
% Cardinality bars
\newcommand{\card}[1]{\ensuremath{|#1|}}


%%STRIKETHROUGH MACRO
\def\str#1{{\setbox1=\hbox{#1}\leavevmode
      \raise.45ex\rlap{\leaders\hrule\hskip\wd1}
      \box1}}
%



\title{Binding and Hierarchy}
\date{Spring 2017}
\author{George Tsoulas}
\institute{Department of Language and Linguistic Science}
 \titlegraphic{\hfill\includegraphics[height=1cm]{../../../../graphics/logo}}

\begin{document}
\maketitle


\begin{frame}
  The model that we have developed so far produces hierarchical structures projected from the lexicon, that conform to the X-bar format and are motivated by two things:
  \begin{itemize}
  \item Local satisfaction of selectional requirements
  \item Constituency tests
  \end{itemize}
\pause
The key word here is \textit{hierarchical}
\end{frame}

\begin{frame}
{The nature of theories}

A crucial aspect of any successful theory is not only that it describes the facts that motivate it well and in detail but, most importantly, that it is able to \textbf{predict} what will happen when different data are encountered which involve the principles of the theory.

\pause

This is what we will test here.

\end{frame}


\begin{frame}
  {Going beyond the basic motivations:  Nominal types}


Languages have a range of nominal expressions (DPs) and a basic property of all these expressions is \textit{Reference}.  In other words, DPs (not all of them, obviously) are able to denote individuals.

\end{frame}

\begin{frame}
  {Classifying DPs from the point of view of reference}

There are three basic classes of DP (when seen from the point of view of their referential properties):

\begin{enumerate}
\item The Pronominal Class \pause
\item The Anaphoric Class \pause
\item The Referential Class \pause
\end{enumerate}

\end{frame}



\begin{frame}
  {Examples}

Consider the different ways we have to refer to an individual:

\ex. 
\a. \terminology{John} came in.
\b. Then, \terminology{He} put on \terminology{his} hat.
\c. \terminology{He} looked at \terminology{himself} in the mirror.
\d. \terminology{The bastard} looked really good in this hat. 

The reasons why languages have different ways to refer to individuals are complex.


\end{frame}

 
\begin{frame}
{Economy of expression}
  \begin{enumerate}
  \item[] When there is no possibility of confusion it may be easier/better to use a reduced expression to make reference to an individual.  A pronoun consists in fact of the features that allow us to identify the relevant individual in the discourse
  \end{enumerate}

However, there seem to be specific restrictions on the use of pronominal expressions.  Thus it is not sufficient that an individual has been made salient on the discourse (in whatever way) in order for the use of a pronominal expression to become licit.


\end{frame}

\begin{frame}
  {Terminology and conventions}

\begin{itemize}
\item We say (of some DPs) that they \textit{refer}. 
\item When two DPs refer to the same individual we say that they are \textit{coreferential} or they corefer.
\item DPs that do not corefer are non-coreferential
\item Reference and Coreference are marked with subscripts on the DPs
\item The DP that a pronoun corefers with is called the pronoun's \textit{antecedent}.
\end{itemize} 
\end{frame}


\begin{frame}
  {Examples}

\ex.
\a. *Himself should decide soon.
\b. *Mary wrote a letter to himself last year.
\c. *John$_i$ hurt him$_i$
\d. *John$_i$ says that Mary$_k$ likes himself$_i$
\e. *Herself$_i$ likes Mary$_j$'s mother$_k$
\f. *He$_i$ heard that [the idiot]$_i$ should win
\b. *He$_i$ saw John$_i$ 

\end{frame}

\begin{frame}
  However much we search for an explanation of these patterns in terms of their meaning or their use in actual discourse we will not find an answer.  It turns out that the patterns depend on the structure of these sentences alone.
\end{frame}



\begin{frame}
  {Anaphors}

Anaphors are elements which do not refer to things independently.  Rather they depend on other elements in order to have their reference fixed.  Most of the elements in the anaphoric class are pronominal.  However, we will reserve the term pronoun/pronominal for personal pronouns like \textit{I, You, He, She etc...}.  We will use the term \textit{Anaphor} for reflexive pronouns: \textit{myself, yourself, himself, herself, ourselves...} and  for reciprocal ones like  \textit{each other}. 
\end{frame}


\begin{frame}
  {Reflexives and Reciprocals}

Both Reflexives and Reciprocals cannot be used unless there is another DP in the sentence that they corefer with.

\ex.  [The girls]$_i$ played with [each other]$_i$

\ex.  Susan$_i$ likes herself$_i$

\ex.  * Herself$_i$ bought chocolate$_i$  

\ex. * [Each other]$_i$ went to the cinema$-i$


\end{frame}


\begin{frame}
  A further contrast is the following:


\ex. 
I met Mary$_i$.  *Herself$_i$ doesn't like me


\ex.
* John$_i$ likes herself$_i$ 

\ex.  
* [The girl]$_i$ likes ourselves$_i$

\ex.
* I$_i$ like herself$_i$ 


\end{frame}



\begin{frame}
  {The generalisation}

The relevant generalisation on the distribution of reflexives is the following:

\begin{block}
{Generalisation on the distribution of reflexives}
A reflexive must:
\begin{enumerate}
\item Be in the same sentence as its antecedent
\item Agree with its antecedent in \textit{Gender, Number, and Person}
\end{enumerate}
\end{block}
\end{frame}


\begin{frame}
  {A Problem}

Given what we said before, why are the following sentences bad?

\ex.
* Himself$_i$ admires Rob

\ex. 
John$_i$'s mother admires himself$_i$

Let us consider the structure of these sentences. 

\end{frame}


\begin{frame}
{The good one}
  \begin{center}
    \begin{forest}
      [TP [DP [Mary]] [T' [T [e]] [VP [V [likes]] [DP [Herself]]]]]
    \end{forest}
  \end{center}
\end{frame}


\begin{frame}
{The bad one}
  \begin{center}
    \begin{forest}
      [TP [DP [DP [John$_i$]] [D' [D ['s]] [NumP [Num] [NP [mother]]]]] [T' [T [e]] [VP [V [likes]] [DP [Himself$_i$]]]]]
    \end{forest}
  \end{center}
\end{frame}

\begin{frame}
  {The generalisation}

The structural  generalisation  is that the reflexive pronoun must appear in a constituent that is inside the sister of its antecedent.  This relationship is captured by the notion of \textit{C-Command} (where C=Constituent)
\end{frame}

\begin{frame}
  {C-Command}

  \begin{block}
    {C-command: Definition}
A node X C-commands a node Y if the sister of X dominates Y
  \end{block}


  \begin{center}
    \begin{forest}
      [[X][Z [\ldots\ldots\ldots Y \ldots\ldots\ldots, triangle]]]
    \end{forest}
  \end{center}

C-command is an extremely important notion that refers to the geometry of the tree.

\end{frame}



\begin{frame}
  {Understanding Reflexives: first try}

We can formulate the following principle (Call it Principle A):

\begin{block}
  {Principle A}
\begin{itemize}
\item The DP antecedent of a reflexive must C-command the reflexive.
\item The reflexive must agree in Person, Number and Gender with its antecedent
\item A reflexive must be coreferential with another DP in the same sentence
\end{itemize}
\end{block}


We will not worry just yet about whether we can unify the three parts into one.
\end{frame}


\begin{frame}
  Consider:


\ex.  John$_i$ believes that Bill$_j$ saw himself$_j$

\ex. * John$_i$ believes that Bill$_j$ saw himself$_k$


\end{frame}

\begin{frame}
  But what about this sentence:


\ex. 
* John$_i$ believes that Bill$_j$ saw himself$_i$


the reason why this sentence is not good is unclear (given the principles that we have so far developed)

\end{frame}


\begin{frame}
  {Distance}

We might think that \textit{John} is too far from the reflexive and that Bill (being closer) intervenes.  The following shows that this is probably the wrong idea:

\ex.
* John$_i$ believes that Mary$_j$ saw himself$_i$

\end{frame}

\begin{frame}
  {A structural proposal}

We can suggest the following:

\ex.  
The reflexive and its antecedent must be in all of the same TPs

This will work for the previous example.  But what about the following:

\ex.
Mary$_i$ noticed John$_j$'s excessive appreciation of himself$_j$

\ex.
* Mary$_i$ noticed John$_j$'s excessive appreciation of herself$_i$

\end{frame}


\begin{frame}
\begin{center}
\scalebox{0.8}{
    \begin{forest}
      [TP [DP [Mary]] [T' [T [e]] [VP [V [noticed]] [DP [DP [John]] [D' [D ['s]] [NumP [Num] [NP[AP [A [excessive]]] [NP[N [appreciation]] [PP [P [of]][DP [herself]]]]]]]]]]]
    \end{forest}
}
\end{center}
\end{frame}

\begin{frame}
  Imagine now that we treat the previous sentence in the same way as :

\ex.
[$_{TP}$ Mary$_i$ noticed that [$_{TP}$ John excessively appreciates herself$_i$]] 

We can give a unified analysis if we suggest the following:

\ex.
The reflexive and its antecedent must be in all the same TPs and DPs.

this does in fact make sense.  Why?

Because TPs and DPs are the only constituents that have subjects.  
\pause
Ok, what does that really mean?

\end{frame}

\begin{frame}
  The question regarding the distribution of reflexives and their potential antecedents is not simply to establish that a reflexive \textit{requires} an antecedent but also how far away, and within what sort of \textit{domain} is the antecedent to be found.  One way to define the relevant domain (given the data) is this:

\begin{block}
{Defining the domain}
The reflexive and its antecedent must be in all the XPs tat have subjects.
\end{block}
  
\end{frame}



\begin{frame}
  {Confirmation}

\ex.
John$_k$ loved [$_{DP}$ the new pictures of himself$_k$]

\ex.
I showed Mary$_k$  [$_{DP}$ several portraits of herself$_k$]


\end{frame}


\begin{frame}
  {Revising the principles on Reflexives}

  \begin{enumerate}
  \item A reflexive must be coreferential with another DP in the same sentence, its antecedent.
  \item A reflexive must agree with its antecedent in person, number, and gender.
  \item The DP antecedent of a reflexive must c-command the reflexive.
  \item The reflexive and its antecedent must be in all the same XPs that have subjects.
  \end{enumerate}


1, 3, 4 above together are refered to as \textit{Principle A of the Binding Theory}

\end{frame}

\begin{frame}
  {Generalising and Reformulating}

Antecedents of reflexives don't have to be names or descriptions like:

\ex. Denis

\ex. The poet who wrote ``Ithaca''

\end{frame}

\begin{frame}

Quantified antecedents are also fine:

\ex.
Every sailor likes himself

\ex.
No girl in the palace dresses herself


These sentences do not mean:

\ex.
Every sailor likes every sailor

\ex.
No girl in the palace dresses no girl in the palace
 
\end{frame}


\begin{frame}

Rather the meanings are:

\ex.
For every x, if x is a sailor, x likes x.

\ex.
For no x, if x is a girl in the palace, x dresses x.

\end{frame}

\begin{frame}
  In these cases it is not right to talk about coreference since, if we say that  \textit{herself} refers to the same individual as \textit{no girl in the palace}, then there must be an individual such that \textit{no girl in the palace} refers to that individual.  But we cannot find such a thing.  The same reasoning applies to \textit{every sailor} and \textit{himself}.

\end{frame}


\begin{frame}
  For these cases we borrow a notion from logic, namely the notion of \textit{variable binding}.  A reflexive/reciprocal will be then the same sort of thing as a logical variable and we will talk about it as being \textit{Bound} by its antecedent.  We have the following definition:


  \begin{block}
    {Bound}
    A DP is \textit{bound} (by its antecedent) just in case there is a C-commanding DP which has the same index.
  \end{block}

\end{frame}


\begin{frame}
{Summarising}
\begin{block}
  {Domain}
The domain of a DP anaphor is the smallest XP with a subject that contains the DP anaphor.
\end{block}
\begin{block}
  {Agreement}
An anaphor must agree with its antecedent in Person, number and gender. (not case)
\end{block}
\begin{block}
  {Principle A}
An anaphor must be bound in its domain.
\end{block}
  
\end{frame}




\end{document}
