\documentclass[10pt]{beamer}
%{article}
%\usepackage{beamerarticle}
\usepackage{tipa}
\usetheme[
%sectiontitleformat=regular,
%everytitleformat=regular,
block=fill,
sectionpage=progressbar,
progressbar=foot,
numbering=fraction,
background=dark]{metropolis}
 
\setbeamercovered{invisible}

%\usecolortheme{owl}
\setbeamertemplate{blocks}[rounded][shadow=true]
\usepackage{booktabs}
\usepackage[scale=2]{ccicons}

\usepackage{pgfplots}
\usepgfplotslibrary{dateplot}


% Packages
\usepackage{linguexsf}
\usepackage{forest}
\usepackage{forest-animate}
\usetikzlibrary{arrows.meta} 
%\usepackage[usenames, dvipsnames]{color}
\usepackage{amsmath}
\usepackage{centernot}

  \tikzset{
    invisible/.style={opacity=0,text opacity=0},
    visible on/.style={alt=#1{}{invisible}},
    alt/.code args={<#1>#2#3}{\alt<#1>{\pgfkeysalso{#2}}{\pgfkeysalso{#3}}}}
\forestset{
  alert on/.style={
    /tikz/alt=<#1>{node options={/tikz/text=alert}}{}},
  dont show before/.style={
%    for first={alert on=#1},
    for ancestors'={
      /tikz/visible on={<#1->},
      edge={/tikz/visible on={<#1->}}},
    for all previous={
      for tree={
        alert on=\number\numexpr#1+1\relax,
        /tikz/visible on={<\number\numexpr#1+1\relax->},
        edge={/tikz/visible on={<\number\numexpr#1+1\relax->}}
      }},
    for children={edge={/tikz/visible on={<#1->}}}}}


\renewcommand{\firstrefdash}{}

% Color math mode!
\everymath{\color[rgb]{.19,.99,.95}}

% Highlight formal text (light blue)
\newcommand{\formal}[1]{\begin{color}[rgb]{.15,.4,.85}#1\end{color}}
% Highlight and bold new terminology (red)
\newcommand{\terminology}[1]{\alert{\textbf{#1}}}
% Translates as (black triple arrow)
\newcommand{\translates}[0]{$\color{black}\Rrightarrow$}
% Set brackets
\newcommand{\set}[1]{\{#1\}}
% Set brackets, shaded
\newcommand{\sett}[1]{\formal{\{#1\}}}
% Set brackets, math mode
\newcommand{\setm}[1]{$\{#1\}$}
% Ordered pair angled brackets
\newcommand{\oset}[1]{\langle #1\rangle}
% Semantic double brackets
\newcommand{\sem}[1]{\ensuremath{\llbracket #1 \rrbracket}}
% Cardinality bars
\newcommand{\card}[1]{\ensuremath{|#1|}}


%%STRIKETHROUGH MACRO
\def\str#1{{\setbox1=\hbox{#1}\leavevmode
      \raise.45ex\rlap{\leaders\hrule\hskip\wd1}
      \box1}}
%



\title{Movement and the EPP}
\date{Spring 2018}
\author{George Tsoulas}
\institute{Department of Language and Linguistic Science}
 \titlegraphic{\hfill\includegraphics[height=1cm]{../../../../graphics/logo}}

\begin{document}
\maketitle

\begin{frame}
  {The Extended Projection Principle}

  \begin{itemize}
  \item The Projection Principle:  Structure is projected from the Lexicon
  \item The Extended Projection Principle:  \textit{Tensed Clauses must have subjects}
  \end{itemize}

\end{frame}


\begin{frame}
  The EPP is one of the reasons why we have raising of embedded subjects to the non-thematic subject position of a verb like \textit{seem}


Equally, however, it has some other consequences and the most important such consequence is its effects on the notion of locality of selection.


\end{frame}

\begin{frame}
  Recall first:
  \begin{itemize}
  \item T selects a VP
  \item T does not select a subject
  \item T does not $\theta$ mark the subject (which is located in [spec TP]
  \end{itemize}

\end{frame}

\begin{frame}


It follows that unless we have a very cumbersome and counter-intuitive definition of selection like:

\begin{block}
  {Locality of selection}
  \begin{itemize}
  \item If $\alpha$ selects $\beta$ as a complement, $\beta$ is the complement of $\alpha$
  \item If $\alpha$ selects $\beta$ as a subject, $\beta$ is the subject of $\alpha$ or of the clause containing $\alpha$
  \item If $\alpha$ selects $\beta$ as an adjunct, $\beta$ is an adjunct to $\alpha$
  \end{itemize}
\end{block}

Then the subject cannot originate in the [spec TP].


\end{frame}

\begin{frame}
  Consider first the following much simpler definition of LoS:

  \begin{block}
    {Locality of Selection}
If a head $\alpha$ selects $\beta$, $\beta$ appears as a complement, subject, or adjunct to $\alpha$ 
  \end{block}

Now this is a much simpler, more elegant and more intuitive definition of LoS.

\end{frame}

\begin{frame}
  What allows us to have this definition is the notion of the EPP.  if the EPP is a feature on T then we can maintain the better definition and propose that:

\ex.
Subjects are always VP (or predicate) internal


\end{frame}

\begin{frame}
  We can see some evidence of this from the phenomenon known as quantifier floating:

\ex.
\a. All the children will leave
\b. The children will all leave

Assuming that the subject is directly generated at TP does not allow us to derive these sentences.
  
\end{frame}


\begin{frame}
  However, if we assume that the subject is generated insider the VP initially and then moves to TP we can analyse these constructions as follows:


  \begin{itemize}
  \item \textbf{Underlying Structure:}  Will [$_{VP}$ [$_{DP}$ all [$_{DP}$ the children]] leave]
  \item \textbf{Whole DP movement:} [$_{DP}$ all [$_{DP}$ the children]] Will\\[1pt]
 [$_{VP}$ \str{[$_{DP}$ all [$_{DP}$ the children]]}  leave]
 \item \textbf{Small DP movement:} [$_{DP}$ the children] Will\\[1pt]
 [$_{VP}$ [$_{DP}$ all [$_{DP}$ \str{the children}]]  leave]
  \end{itemize}

\end{frame}



\begin{frame}
  This idea also allows us to maintain the locality of thematic assignment, which is also a welcome result.
\end{frame}



\begin{frame}
  {Taking Stock}

  \begin{itemize}
  \item A sentence as it is pronounced has two basic structures.
    \begin{itemize}
    \item An underlying structure
    \item A surface structure
    \item The underlying and surface structures are related through a series of other intermediate structures.  (Note here that when we say structure we always mean a tree).  Some of the intermediate trees are partial.
    \end{itemize}
\item  Building the underlying structure involves selecting lexical items and arranging them into constituents
\item There are two basic operations:
  \begin{itemize}
  \item Merge: puts together lexical items and derived objects to create more complex objects
  \item Move: moves a previously selected and merged element to a different position essentially \textit{remerging it}.
  \end{itemize}

  \end{itemize}

\end{frame}

\begin{frame}
  Notice that selectional restrictions can be satisfied in two ways:

  \begin{itemize}
  \item By Merge, as in the case of complements and adjuncts (although we set adjuncts aside for now)
  \item By Move as in the case of EPP phenomena and other phrasal movement but also in the case of head movements where the affixal nature of some heads can be seen as a selectional requirement.
  \end{itemize}
\end{frame}


\end{document}
