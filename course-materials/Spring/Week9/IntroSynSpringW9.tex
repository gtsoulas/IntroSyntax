%%% IntroSynSpringW9.tex --- 
%% 
%% Filename: IntroSynSpringW9.tex
%% Description: 
%% Author: George Tsoulas
%% Maintainer: 
%% Created: Fri Feb 26 12:44:34 2016 (+0000)
%% Version: 
%% Package-Requires: ()
%% Last-Updated: 
%%           By: 
%%     Update #: 0
%% URL: 
%% Doc URL: 
%% Keywords: 
%% Compatibility: 
%% 
%%%%%%%%%%%%%%%%%%%%%%%%%%%%%%%%%%%%%%%%%%%%%%%%%%%%%%%%%%%%%%%%%%%%%%
%% 
%%% Commentary: 
%% 
%% 
%% 
%%%%%%%%%%%%%%%%%%%%%%%%%%%%%%%%%%%%%%%%%%%%%%%%%%%%%%%%%%%%%%%%%%%%%%
%% 
%%% Change Log:
%% 
%% 
%%%%%%%%%%%%%%%%%%%%%%%%%%%%%%%%%%%%%%%%%%%%%%%%%%%%%%%%%%%%%%%%%%%%%%
%% 
%% This program is free software: you can redistribute it and/or modify
%% it under the terms of the GNU General Public License as published by
%% the Free Software Foundation, either version 3 of the License, or (at
%% your option) any later version.
%% 
%% This program is distributed in the hope that it will be useful, but
%% WITHOUT ANY WARRANTY; without even the implied warranty of
%% MERCHANTABILITY or FITNESS FOR A PARTICULAR PURPOSE.  See the GNU
%% General Public License for more details.
%% 
%% You should have received a copy of the GNU General Public License
%% along with GNU Emacs.  If not, see <http://www.gnu.org/licenses/>.
%% 
%%%%%%%%%%%%%%%%%%%%%%%%%%%%%%%%%%%%%%%%%%%%%%%%%%%%%%%%%%%%%%%%%%%%%%
%% 
%%% Code:

\documentclass[10pt]{beamer}
%{article}
%\usepackage{beamerarticle}
\usepackage{tipa}
\usetheme[
%sectiontitleformat=regular,
%everytitleformat=regular,
block=fill,
sectionpage=progressbar,
progressbar=foot,
numbering=fraction,
background=dark]{metropolis}
 
\setbeamercovered{invisible}

%\usecolortheme{owl}
\setbeamertemplate{blocks}[rounded][shadow=true]
\usepackage{booktabs}
\usepackage[scale=2]{ccicons}

\usepackage{pgfplots}
\usepgfplotslibrary{dateplot}


% Packages
\usepackage{linguexsf}
\usepackage{forest}
\usepackage{forest-animate}
\usetikzlibrary{arrows.meta} 
%\usepackage[usenames, dvipsnames]{color}
\usepackage{amsmath}
\usepackage{centernot}

  \tikzset{
    invisible/.style={opacity=0,text opacity=0},
    visible on/.style={alt=#1{}{invisible}},
    alt/.code args={<#1>#2#3}{\alt<#1>{\pgfkeysalso{#2}}{\pgfkeysalso{#3}}}}
\forestset{
  alert on/.style={
    /tikz/alt=<#1>{node options={/tikz/text=alert}}{}},
  dont show before/.style={
%    for first={alert on=#1},
    for ancestors'={
      /tikz/visible on={<#1->},
      edge={/tikz/visible on={<#1->}}},
    for all previous={
      for tree={
        alert on=\number\numexpr#1+1\relax,
        /tikz/visible on={<\number\numexpr#1+1\relax->},
        edge={/tikz/visible on={<\number\numexpr#1+1\relax->}}
      }},
    for children={edge={/tikz/visible on={<#1->}}}}}


\renewcommand{\firstrefdash}{}

% Color math mode!
\everymath{\color[rgb]{.19,.99,.95}}

% Highlight formal text (light blue)
\newcommand{\formal}[1]{\begin{color}[rgb]{.15,.4,.85}#1\end{color}}
% Highlight and bold new terminology (red)
\newcommand{\terminology}[1]{\alert{\textbf{#1}}}
% Translates as (black triple arrow)
\newcommand{\translates}[0]{$\color{black}\Rrightarrow$}
% Set brackets
\newcommand{\set}[1]{\{#1\}}
% Set brackets, shaded
\newcommand{\sett}[1]{\formal{\{#1\}}}
% Set brackets, math mode
\newcommand{\setm}[1]{$\{#1\}$}
% Ordered pair angled brackets
\newcommand{\oset}[1]{\langle #1\rangle}
% Semantic double brackets
\newcommand{\sem}[1]{\ensuremath{\llbracket #1 \rrbracket}}
% Cardinality bars
\newcommand{\card}[1]{\ensuremath{|#1|}}


%%STRIKETHROUGH MACRO
\def\str#1{{\setbox1=\hbox{#1}\leavevmode
      \raise.45ex\rlap{\leaders\hrule\hskip\wd1}
      \box1}}
%



\title{Control and Raising}
\date{Spring 2017}
\author{George Tsoulas}
\institute{Department of Language and Linguistic Science}
 \titlegraphic{\hfill\includegraphics[height=1cm]{../../../../graphics/logo}}

\begin{document}
\maketitle



\begin{frame}

  We have seen so far a number of cases of movement that involved infinitival clauses.  In this lecture we take a closer look at infinitives and some of the issues surrounding them.

\end{frame}
{the main idea}

The main question is to understand the differences between the following cases:

\ex.
Mary tried to sleep

and 

\ex.
Mary seems to like pastrami 



Let's start by comparing them with respect to what we know already.


\begin{frame}


The first thing that we know about sentences with verbs like \textit{seem} is that the surface subject is selected \textit{only} by the embedded verb.  In other words, if the subject is incompatible with the embedded verb the sentence is ungrammatical:

\ex.
* The pencil seems to devour the cheese

\ex.
The pencil seems to write well


The verb seems does not impose any restrictions.

  
\end{frame}


\begin{frame}
  Compare now to what happens with a verb like \textit{Try}

\ex.
* The pencil tried to devour the cheese

\ex.
* The pencil tried to write well


It looks like in order to have a grammatical sentence with the verb try then its subject must satisfy the semantic restrictions of try (it must be able to be an agent)

So:  try $\neq$ seem 

\end{frame}

\begin{frame}

\ex.
Bill tried to devour the cheese

\ex.
Bill tried to write well

But:

\ex.
* Bill tried to elapse

What this shows is that the subject of TRY must \textit{also} be compatible with the embedded verb

\end{frame}

\begin{frame}


We can summarise the pattern so far using the notion of selection:

\begin{itemize}
\item In the case of \textit{Seem} the surface subject is only selected by the embedded verb.
\item In the case of \textit{Try} the surface subject is selected both by the main verb (try) and the embedded one.  
\end{itemize}

\pause

How can this be possible ?
  
\end{frame}

\begin{frame}
  The problem is that both locality of selection and the $\theta$ criterion explicitly disallow long-distance selection on the one hand and bearing two $\theta$-roles at the same time.
\end{frame}

\begin{frame}
  Our only way out here is to assume that the embedded clause actually has its own subject, which is phonologically null, silent.

Let's assume that to be true.


\end{frame}


\begin{frame}
  
The question now is what is the nature of this silent subject.

Two options:

\begin{itemize}
\item The embedded subject is an exact unpronounced copy of the matrix subject
\item The embedded subject is some kind of pronominal or anaphoric expression. 
\end{itemize}


\end{frame}


\begin{frame}
  Fortunately we can discard the first option relatively easily.

Consider what this would amount to when the subject of the matrix is a word like \textit{everyone}:


\ex.
Everyone wants to be famous

does this mean:

\ex.  Everyone wants everyone to be famous

or

\ex.
Everyone wants him/herself to be famous


 
\end{frame}







\begin{frame}
  Clearly, the meaning of the sentence is not that everyone wants himself AND everyone else to be famous.  Rather, the meaning is that where a statement is made about everyone's desires about themselves. 
\end{frame}

\begin{frame}
  The conclusion from this is that we cannot have an identical copy as the subject of the infinitive. 
 

So how about the case of silent pronoun or a reflexive?
\end{frame}


\begin{frame}
  Consider the following examples:

\ex.  
Only Churchill remembered [ [???] giving the Blood Toil Sweat and Tears speech]

\ex.  
Only Churchill remembered [ himself giving the Blood Toil Sweat and Tears speech]

\ex.  
Only Churchill remembered [ Churchill  giving the Blood Toil Sweat and Tears speech]

\ex.  
Only Churchill remembered [ his giving the Blood Toil Sweat and Tears speech]

\end{frame}

\begin{frame}

  The interest of such examples can be seen when we consider what happens in case there is some other person who has actually heard the speech on the radio and therefore also remembers it. Which one of the above sentences is true in that case?

\end{frame}



\begin{frame}
  In that case, clearly:

\ex.  
Only Churchill remembered [ Churchill  giving the Blood Toil Sweat and Tears speech]

\ex.  
Only Churchill remembered [ his giving the Blood Toil Sweat and Tears speech]


Are false.  Therefore we confirm that it cannot be a silent name or a silent pronoun like \textit{his}.

\end{frame}

\begin{frame}
  How about :

\ex.  
Only Churchill remembered [ himself giving the Blood Toil Sweat and Tears speech]


Here judgements vary.  We can take this as an indication that a silent reflexive cannot be the whole story for all cases.


\end{frame}




\begin{frame}
  It is in fact cases such as these that have led to the postulation of a special pronominal anaphor, which is called PRO, that fills the subject position of infinitival clauses.


Pro is said to be \textit{Controlled}  by a C-commanding DP in the main clause and this DP is either the subject or the object of the matrix verb.
  
\end{frame}

\begin{frame}
  If the subject of the matrix controls PRO then the the matrix verb is called
  \begin{itemize}
  \item A subject control verb.  (Try, want, attempt, hope)
  \end{itemize}
If it is the object that controls PRO we are talking about :
\begin{itemize}
\item An object control verb (persuade, order...)
\end{itemize}

\end{frame}



\begin{frame}
  The important message from this is:

  \begin{itemize}
  \item There are different types of referential dependencies (Binding, Control, Accidental co-reference)
  \item There are different types of infinitival clauses, namely:
    \begin{itemize}
    \item Raising 
    \item Control
    \end{itemize}
  \end{itemize}

Whether we are looking at a case of raising or a case of control depends on the matrix verb and whether it selects and $\theta$-marks its subject.



\end{frame}



\end{document}




%%%%%%%%%%%%%%%%%%%%%%%%%%%%%%%%%%%%%%%%%%%%%%%%%%%%%%%%%%%%%%%%%%%%%%
%%% IntroSynSpringW9.tex ends here
