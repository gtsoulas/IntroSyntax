\documentclass[12pt]{article}
 %\textwidth=7in
%\textheight=9.5in
%\topmargin=-1in
%\headheight=0in
%\headsep=.5in
%\hoffset  -.85in
%\usepackage{fullpage}
%\pagestyle{empty}
\usepackage{lastpage}
\usepackage{timestamp}
\usepackage{ifthen}
%\usepackage[a4paper, total={5in, 7in}]{geometry}
\usepackage{fancyhdr}    
\usepackage{enumerate}

\pagestyle{fancy}


\rhead{\ifthenelse{\value{page}=1}{}{2015-2016}}
\lhead{\ifthenelse{\value{page}=1}{}{\textit{Introduction to Syntax}}}
 \lfoot{\texttt{\timestamp}}
 \rfoot{\thepage/\pageref{LastPage}}
 \cfoot{}

\renewcommand{\thefootnote}{\fnsymbol{footnote}}
\begin{document}


\begin{center}
{\bf L11C Introduction to Syntax\footnote{This document may be slightly updated.  The most up-to-date version will be on the VLE.  You will receive an email alerting you to any changes.}\\
2015 - 2016}
\end{center}

\setlength{\unitlength}{1in}

\begin{picture}(6,.1) 
\put(-.27,0) {\line(1,0){5}}         
\end{picture}

 

\renewcommand{\arraystretch}{2}

\vskip.25in
\noindent\textbf{Instructor:} George Tsoulas, \begin{tabular}[t]{ll}email:& george.tsoulas@york.ac.uk\\[-4mm] Office: & V/C/208 \\[-4mm] Phone: & 322664 \end{tabular}

\noindent\textbf{Seminar Instructors:}
\begin{tabular}[t]{ll}Nikki Chen & zc643@york.ac.uk \\[-3mm] Miyuki Kamiya & mn614@york.ac.uk \\[-3mm] Aiqing Wang & aiqing.wang@york.ac.uk \\[-3mm] Norman Yeo & norman.yeo@york.ac.uk\\ \end{tabular}
\vskip.25in
\noindent\textbf{Office Hours:} Tuesday: 9.30 - 11.00 and by  appointment.\footnote{Your seminar tutors will let you know about their office hours.}

\vskip.25in
\noindent\textbf{Textbook:}  Dominique Sportiche, Hilda Koopman, and Edward Stabler (2014): \textit{An Introduction to Syntactic Analysis and Theory}, Wiley Blackwell.

\vskip.25in

\noindent 
\section{Course Outline} 

This module is an introduction to syntax.  Syntax is the scientific study of the structural organisation of sentences. Syntactic theory provides a model of the tacit knowledge that speakers have of the syntactic structure of their language and separates that which is part of \textit{Universal Grammar} (UG) and that which varies between languages.  The theory of the contents of UG is the theory of the Language Faculty, the shared, species specific genetic endowment that makes language acquisition possible.  The theory of variation among languages tells us what is the scope of the variation and which linguistic properties are susceptible to variation.      

\subsection{Course Programme}
\subsubsection*{Autumn Term}

%\begin{center} 
%\begin{minipage}{6in}
%\begin{flushleft}
\noindent \textit{Week 2} \dotfill \begin{tabular}[t]{l} General Introduction\\[-4mm] The scientific study of Syntax\\[-4mm] The notion of structure\\[-4mm] Structure in words, morphemes\\ \end{tabular}
\begin{flushright}
\begin{tabular}[t]{|l|p{2in}|}\hline \textbf{Reading} & Chapter 1 and Chapter 2 up to page 28 \\\hline \end{tabular}
\end{flushright}
\textit{Week 3} \dotfill \begin{tabular}[t]{l} Structure in compounds\\[-4mm] Headedness\\[-4mm] Exceptions. \end{tabular}
\begin{flushright}
\begin{tabular}[t]{|l|p{2in}|}\hline \textbf{Reading} & The rest of chapter 2 (28-41) \\\hline \end{tabular}
\end{flushright}
\textit{Week 4} \dotfill \begin{tabular}[t]{l} Word Order and Constituency I \end{tabular}
\begin{flushright}
\begin{tabular}[t]{|l|p{2in}|}\hline \textbf{Reading} & Chapter 3.  pp. 43-58 \\\hline \end{tabular}
\end{flushright}
\textit{Week 5} \dotfill \begin{tabular}[t]{l} Word Order and Constituency II. \end{tabular}
\begin{flushright}
\begin{tabular}[t]{|l|p{2in}|}\hline \textbf{Reading} & The rest of chapter 3 (58-85) \\\hline \end{tabular}
\end{flushright}
\textit{Week 6} \dotfill \begin{tabular}[t]{l} Reading Week \end{tabular}
\begin{flushright}
\begin{tabular}[t]{|l|p{2in}|}\hline \textbf{Reading} & Review in detail the first three chapters \\\hline \end{tabular}
\end{flushright}
\textit{Week 7} \dotfill \begin{tabular}[t]{l} Clauses \end{tabular}
\begin{flushright}
\begin{tabular}[t]{|l|p{2in}|}\hline \textbf{Reading} & Chapter 4 \\ \hline \end{tabular}
\end{flushright}
\textit{Week 8} \dotfill \begin{tabular}[t]{l} Other Phrases \end{tabular}
\begin{flushright}
\begin{tabular}[t]{|l|p{2in}|}\hline \textbf{Reading} & Chapter 5 \\\hline \end{tabular}
\end{flushright}
\textit{Week 9} \dotfill \begin{tabular}[t]{l} X-bar theory I\\ \end{tabular}
\begin{flushright}
\begin{tabular}[t]{|l|p{2in}|}\hline \textbf{Reading} & Chapter 6 pp. 127-140 \\\hline \end{tabular}
\end{flushright}
\textit{Week 10} \dotfill \begin{tabular}[t]{l} X-bar theory II. \end{tabular}
\begin{flushright}
\begin{tabular}[t]{|l|p{2in}|}\hline \textbf{Reading} & Chapter 6.  pp. 140 - 155. \\\hline \end{tabular}
\end{flushright}

%\vskip.25in


\subsubsection*{Christmas Break}  As the break is quite long I expect to you to re-read at least once the material covered in the Autumn term and do the associated exercises.

%\vskip.25in

\subsubsection*{Spring Term}
\textit{Week 2} \dotfill \begin{tabular}[t]{l} Review of Autumn term material\\ \end{tabular} \\
\textit{Week 3} \dotfill \begin{tabular}[t]{l} Binding and Anaphora I \end{tabular}
\begin{flushright}
\begin{tabular}[t]{|l|p{2in}|}\hline \textbf{Reading} & Chapter 7 (157-172) \\\hline \end{tabular}
\end{flushright}
\textit{Week 4} \dotfill \begin{tabular}[t]{l} Binding and Anaphora II \end{tabular}
\begin{flushright}
\begin{tabular}[t]{|l|p{2in}|}\hline \textbf{Reading} & Chapter 7.  pp. 174-186 \\\hline \end{tabular}
\end{flushright}
\textit{Week 5} \dotfill \begin{tabular}[t]{l} Selection and Locality \end{tabular}
\begin{flushright}
\begin{tabular}[t]{|l|p{2in}|}\hline \textbf{Reading} & Chapter 8 (187-208) \\\hline \end{tabular}
\end{flushright}
\textit{Week 6} \dotfill \begin{tabular}[t]{l} Reading Week \end{tabular}
\begin{flushright}
\begin{tabular}[t]{|l|p{2in}|}\hline \textbf{Reading} & Review in detail chapters 7 and 8 (including a first pass at the part that we did not yet look at) \\\hline \end{tabular}
\end{flushright}
\textit{Week 7} \dotfill \begin{tabular}[t]{l} Phrasal Movements \end{tabular}
\begin{flushright}
\begin{tabular}[t]{|l|p{2in}|}\hline \textbf{Reading} & Chapter 8 (210-238) \\ \hline \end{tabular}
\end{flushright}
\textit{Week 8} \dotfill \begin{tabular}[t]{l} Raising and Control \end{tabular}
\begin{flushright}
\begin{tabular}[t]{|l|p{2in}|}\hline \textbf{Reading} & Chapter 9 \\\hline \end{tabular}
\end{flushright}
\textit{Week 9} \dotfill \begin{tabular}[t]{l} More on raising and control\\ \end{tabular}
\begin{flushright}
\begin{tabular}[t]{|l|p{2in}|}\hline \textbf{Reading} & Chapter 6 pp. 127-140 \\\hline \end{tabular}
\end{flushright}
\textit{Week 10} \dotfill \begin{tabular}[t]{l} Review of the overall theory \end{tabular}
%\begin{flushright}
%\begin{tabular}[t]{|l|p{2in}|}\hline \textbf{Reading} & \\\hline \end{tabular}
%\end{flushright}

\subsubsection*{Easter Break}  Read the everything from the beginning until the end of chapter 9.


\subsubsection*{Summer Term:} During the summer term you will have revision sessions.  More details on the organisation of these sessions will be made available at a later point.


\section{Other information}


\subsection{Lectures and Seminars}
 Each week you will have a one hour lecture and a one hour seminar.  The lecture will present and discuss the material covered in the relevant part of the textbook.  The lecture will also serve to give you some more general perspectives going beyond the textbook.  The seminars aim at consolidating your knowledge by reviewing fundamental concepts introduced in the lecture, discussing larger sets of data, and going through exercises.

\subsubsection*{Preparing for Lectures}  The best way to prepare for lectures is to do the reading before the lecture.  This is what I will generally expect.

\subsubsection*{Preparing for Seminars} It is more important that you arrive prepared at a seminar than at a lecture.  A lecture is meant to introduce new material.  A seminar to review that material.  So in order to get the most out of you seminars you should arrive having done the following:
\begin{itemize}
\item The reading assigned for the week (again).  On average this is no more that 15 pages so nothing daunting really.
\item You must have done the exercises and you should bring along to the seminar your efforts (whether completed or not).
\item You should have prepared a list of questions (it does not have to be a long list). 
\item You should be prepared to participate in the discussion, it is an essential part of a seminar session.
\end{itemize}

\subsubsection*{VLE}
There is a VLE (Virtual Learning Environment) site associated with this module.  All materials will be posted there and will be available in good time.
\subsubsection*{Lecture Capture}
Lectures for this module will be recorded using the Replay system, which allows you to listen to the audio of the lecture again afterwards, and view anything that was shown on the computer screen during the lecture. This is a response to feedback from past students on other modules who found such recordings very useful e.g. for revision and review.
However, please note that
\begin{enumerate}[i]
\item We may sometimes decide not to make the recording for a particular lecture available, for copyright, technical, or other  reasons, and
\item Not all visual components of a lecture will be recorded e.g. if the lecturer uses the blackboard during the lecture. For these reasons you must not use the recordings as a replacement for attending lectures in person, other than in cases of e.g. illness.
\end{enumerate}

\subsection{Homework and Assessment} Doing the readings is part of your homework.  There will also be two other types of homework:
\subsubsection*{Formative Assessment} These are the exercises that will be set each week and you need to do and bring to the seminars where they will be discussed so that see what you did right and what needs more work.  
There is also a formative \textit{mock} exam taking place in week 1 of the spring term.
\paragraph{Submitting Formative Assessment} You must prepare the exercises for the seminar.  After you have discussed it in the seminar, you must submit the exercises \textit{together with a short self assessment}.  You do not need to give a mark to your work, but to tell me in a sentence or two what you did right and what wrong (And why if you know).  These will only be returned to you at the end of the year so it will be a very good idea to keep a copy.  They are to be submitted in George's metal box on the Tuesday following each seminar.  You submit those with your name on and also you must state the name of your seminar tutor.
\subsubsection*{Summative Assessment} This is work that you should do on your own and hand in anonymously (this means that you should only put your exam number and not your name).  Summative assessment \textit{counts} towards the final mark for your module.  There are two types of summative assessment:
\begin{itemize}
\item Exercises that you submit throughout the module (see below for dates). These count for 40\% of the final mark.
\item A 90mn closed exam at the end of the module in Term 3.  The exam counts for 60\% of the final module mark. 
\end{itemize}
\subsubsection*{Feedback} You will receive feedback on your work in two forms.
\begin{itemize}
\item Discussion in Seminars
\item Model Answers to exercises
\end{itemize}
\subsubsection*{Submission times}
Summative assessment is to be submitted at the following times Submission is again in George's metal box):
\begin{itemize}
\item Autumn term
  \begin{itemize}
  \item Thursday 12 November 2015 at 12.00 noon.  (Week 7)
  \end{itemize}
\item Spring Term
  \begin{itemize}
  \item Thursday 7 January 2016 at 12.00 noon (Week 1)
  \item Thursday 18 February 2016 at 12.00 noon (Week 7)
  \end{itemize}
\item Summer Term
  \begin{itemize}
  \item Thursday 14 April 2016 at 12.00 noon (Week 1)
  \end{itemize}

\end{itemize}

\subsection{Reassessment}
This module is reassessed at the module-level. This means that if you are required to resit this module, you will have to take the following assessment:
\begin{itemize}
\item Closed exam (90 minutes) - 100\%
\end{itemize}


\end{document}
